\toadd{Especially if I do take out the related work, I should add a short
intro to why Dissent is a good thing to use}
Our protocol uses this to elect servers for an instance of Dissent in
Numbers \cite{din}, which can then be used for latency-sensitive applications
without sacrificing the superior security properties of Hardened Dissent.

The primary limitation of this
version is performance --- there is a detailed security analysis demonstrating
the anonymity preserving properties of this layer, and so we use it as a
fallback layer for reconfiguring if servers misbehave by dropping out, and for
initial setup of the Dissent in Numbers (DIN) layer.

    A map \texttt{Committees} mapping names of elected statuses to sets of
      elements of the \KwRoster. For example, \texttt{Committees} might
      consist of the key ``\texttt{servers}'', with the value $\{A, B, C\}$,
      where $A, B,$ and $C$ are the participants listed in $R$ which have been
      elected to act as \texttt{servers} in an instance of Dissent in
      Numbers.\todosubst{Obsolete. Make generic.}

  Dissent is an alternative to Tor that provides provable anonymity even if
  only one server in the network is honest\cite{p2pd}.
  \section{Protocol Explanation}
  In its present form, a Dissent cluster consists of $m$ servers and $n$
  connected clients\cite{din}. Provable anonymity is
  achieved through a modified version of the Dining Cryptographers
  problem\cite{chaum_dining_1988}: each client $i$ shares a secret $K_{ij}$
  with each server $j$. Communication proceeds in rounds, within which each
  client has a designated $k$-bit slot.  Before any messages are sent, a
  secure shuffle\cite{neff} assigns each client to a slot so
  that the owner of a slot is the only node in the system which knows who owns
  that slot.  In any client $s$'s slot, every client and every server
  generates $k$ bits of random noise seeded with each of its shared secrets
  $K_{ij}$, and combines these with an exclusive or (xor) operation to produce
  that node's ciphertext. Client $s$ also combines (via xor) a $k$-bit message
  with its noise to create its ciphertext. The combination (via xor) of all
  clients' and servers' ciphertext includes the noise stream associated with
  each shared secret twice, and so all noise cancels out and client $s$'
  message is revealed. However, since deciphering this requires the
  participation of all nodes in the system, it is impossible to tell which
  client transmitted a message in a given slot. Dissent also incorporates an
  accountability mechanism, allowing any node that disrupts the protocol to be
  detected and removed from the cluster
  \cite{verdict}.

  The original Dissent was fully peer-to-peer
  \cite{p2pd}. The shift to a client-server model
  allows for significantly improved performance, but it introduces several new
  concerns, particularly relating to misbehaving servers, a new class of DoS
  attacks, and group formation.

  \section{Properties}
  \begin{theorem} At the end of round $i$, \ldots \todo{semicomputability}\end{theorem}
  \begin{theorem} For any round $i$, all participants know the results of round
    $i$ before any know the results of round
    $i+1$.\end{theorem}\label{theorem:rounds}
  % todo: put this somewhere
  % \paragraph{Accountability:} In the context of Dissent, accountability refers
  % to the ability of a protocol to detect and exclude participants who disrupt
  % the protocol \cite{sec}, while proving that the disruptor did
  % indeed disrupt the protocol. Such a mechanism is necessary in peer-to-peer
  % protocols like Dining Cryptographers in which a single disruptor can make
  % the result of a round unusable. In Dissent, accountability checks occur
  % without revealing the link between any message and its sender --- moreover,
  % it is not possible to deliberately exclude a participant on the basis of
  % valid messages the participant sends. That is, the Dissent accountability
  % mechanism does not break anonymity.
\subsection{Dissent in Numbers}
An instance of Dissent in Numbers consists of $\NumClients$ clients and
$\NumServers$ servers.  Communication takes place in \emph{rounds}, wherein each
client has an opportunity to broadcast a message to the entire client set.
Assuming $\NumHonest$ of the $\NumClients$ clients are honest, each client is
guaranteed that, at the protocol level, its message will be anonymous among the
$\NumHonest$ honest clients unless all servers collude with each
other.\toadd{Move the security properties to the goals section and only describe
the functionality here} Since all clients receive each client's messages in a
deterministic order, there is a well-defined sequence of rounds which we can
associate with a monotonically increasing \emph{round ID}.

\subsection{Dissent-in-Numbers Layer}
\todo[inline]{TODO: Describe uses of DIN}
In its current form, Dissent in Numbers lacks the ability to handle server
faults, and it depends on a pre-existing well-known set of servers. However, it
also provides dramatically superior performance as compared with Hardened
Dissent. Our protocol adds group management, fault tolerance, and some forms of
censorship resistance to Dissent in Numbers, providing a framework for a more
versatile system.

\missingfigure{TODO: Draw the topology and a flow chart showing pseudonyms/public
keys}

