Anonymous communication significantly constrains the ability of oppressive
regimes and vigilante groups alike to suppress dissent. Newly available
information about global-scale surveillance in today's
centralized internet infrastructure has rendered a swath of anonymity tools
vulnerable.

A trustworthy anonymity tool in the post-Snowden era must be resilient to both
surveillance and censorship: It should guarantee its users' anonymity even in
the face of a global passive adversary, and it should be unrealistic for such an
adversary to simply prevent users from accessing it. A useful anonymity tool
must also perform with reasonably low latency - a property which often trades
off with security and availability.

% TODO: integrate
In its current form, Dissent in Numbers lacks the ability to handle server
faults, and it depends on a pre-existing well-known set of servers. However, it
also provides dramatically superior performance as compared with Hardened
Dissent.
Our protocol can be used to elect servers for an instance of Dissent in
Numbers \cite{din}, which can then be used for latency-sensitive applications
without sacrificing the superior security properties of Hardened Dissent.


\section{Background: Dissent in Numbers}
  Peer-to-peer Dissent provides stronger anonymity guarantees than Tor, but at
  the cost of significantly degraded performance. A new version of the
  protocol called Dissent in Numbers \cite{din} aims to provide similarly
  strong anonymity properties at scale. For a single sender to transmit a 128
  kB message to a 16-member group under \cite{p2pd} requires more than a
  minute; under \cite{din} this takes less than 5 seconds.

  Dissent in Numbers achieves this performance by using a $\NumClients$
  clients/$\NumServers$ server model.
  Rather than requiring $ \NumClients^2$ shared secrets (one for each pair of
  clients), Dissent in Numbers uses only one shared secret per server/client
  pair, for a total of $\NumServers\NumClients$. Instead of communicating
  directly with every other node, each client comunicates only with every
  server in each round.

  Dissent in Numbers provides anonymity among the honest clients so long as at
  least one of the servers is honest. This means that in a system with
  $\NumHonest$ clients out of $\NumClients$, peer to peer Dissent can be shown
  to always preserve $\NumHonest$-anonymity among the honest clients, whereas
  in Dissent in Numbers, only $\NumServers$ need be compromised, regardless of
  the number of honest clients. Further, peer-to-peer Dissent is able to
  respond to changes in the topology of the group, including the case where
  one or more members is removed due to misbehavior. In Dissent in Numbers,
  the servers are assumed to always be reliable, and while clients can detect
  server misbehavior, there is no built-in mechanism to respond to this while
  continuing to use the protocol.

  % Relays
  Even where a sufficient number of servers are honest and non-disruptive,
  Dissent in Numbers introduces additional vulnerability to censorship.
  One potential approach to making Dissent widely available would be to have
  well-known, globally dispersed Dissent servers available for clients to
  connect to, similar to the current state of Tor. Any such well-known server
  list, however, is susceptible to blocking by internet service providers. It
  would therefore be preferable to have servers be short-lived, or at least
  not well known. Since Dissent takes place over regular TCP connections,
  detecting that the protocol is being executed without knowledge of the
  addresses of servers would be difficult to accomplish without a great number
  of false positives \cite{houmansadr_parrot_2013}, so this may be enough to
  realistically preclude most attempts to block access to the protocol
  entirely.

  We now show how the protocol presented in this paper can be used to
  bootstrap Dissent in Numbers, so that in the usual case clients can achieve
  its superior performance, but in error cases the cluster can recover from
  these sorts of faults. Since our protocol is suitable for execution at the
  level of a local cluster, it further extends the advantages of Dissent in
  Numbers to environments where server identities must be ephermeral to avoid
  censorship.

\section{Anonymous Petitions for Group Management in Dissent}
  % intro to ours
  Our protocol enables clusters of Dissent clients to elect temporary
  servers among themselves, allowing servers to either step down (e.g., by
  going offline) or be impeached by some portion of the clients.

  We assume a cluster of $\NumClients$ peers connected by TCP connections. To
  launch an instance of Dissent in Numbers, they first launch an instance of our
  protocol. The \KwManifest should contain a \KwServer s field, containing the
  subset of \KwRoster members currently acting as servers. This field may
  initially be blank, or it may be specified offline.

  Once a server set is spefified, all participants have all information they
  need to launch a Dissent in Numbers instance \cite{din}. Initially, and
  whenever a vote to change the \KwManifest~passes at the Anonymous Petition
  layer, the \KwMember s newly designated as servers begin running server
  instances of Dissent in Numbers, and all \KwMember s (including the ones now
  running servers) begin running client instances.

  By using this protocol to handle group membership, server election, and fault
  tolerance, we bring resiliance, fault tolerance, and some forms of censorship
  resistance to Dissent in Numbers, providing a framework for a more versatile
  system.

% The primary limitation of this
% version is performance --- there is a detailed security analysis demonstrating
% the anonymity preserving properties of this layer, and so we use it as a
% fallback layer for reconfiguring if servers misbehave by dropping out, and for
% initial setup of the Dissent in Numbers (DIN) layer.
%
%     A map \texttt{Committees} mapping names of elected statuses to sets of
%       elements of the \KwRoster. For example, \texttt{Committees} might
%       consist of the key ``\texttt{servers}'', with the value $\{A, B, C\}$,
%       where $A, B,$ and $C$ are the participants listed in $R$ which have been
%       elected to act as \texttt{servers} in an instance of Dissent in
%       Numbers.\todosubst{Obsolete. Make generic.}
%
%   Dissent is an alternative to Tor that provides provable anonymity even if
%   only one server in the network is honest\cite{p2pd}.
%   \section{Protocol Explanation}
%   In its present form, a Dissent cluster consists of $m$ servers and $n$
%   connected clients\cite{din}. Provable anonymity is
%   achieved through a modified version of the Dining Cryptographers
%   problem\cite{chaum_dining_1988}: each client $i$ shares a secret $K_{ij}$
%   with each server $j$. Communication proceeds in rounds, within which each
%   client has a designated $k$-bit slot.  Before any messages are sent, a
%   secure shuffle\cite{neff} assigns each client to a slot so
%   that the owner of a slot is the only node in the system which knows who owns
%   that slot.  In any client $s$'s slot, every client and every server
%   generates $k$ bits of random noise seeded with each of its shared secrets
%   $K_{ij}$, and combines these with an exclusive or (xor) operation to produce
%   that node's ciphertext. Client $s$ also combines (via xor) a $k$-bit message
%   with its noise to create its ciphertext. The combination (via xor) of all
%   clients' and servers' ciphertext includes the noise stream associated with
%   each shared secret twice, and so all noise cancels out and client $s$'
%   message is revealed. However, since deciphering this requires the
%   participation of all nodes in the system, it is impossible to tell which
%   client transmitted a message in a given slot. Dissent also incorporates an
%   accountability mechanism, allowing any node that disrupts the protocol to be
%   detected and removed from the cluster
%   \cite{verdict}.
%
%   The original Dissent was fully peer-to-peer
%   \cite{p2pd}. The shift to a client-server model
%   allows for significantly improved performance, but it introduces several new
%   concerns, particularly relating to misbehaving servers, a new class of DoS
%   attacks, and group formation.
%
%   \section{Properties}
%   \begin{theorem} At the end of round $i$, \ldots \todo{semicomputability}\end{theorem}
%   \begin{theorem} For any round $i$, all participants know the results of round
%     $i$ before any know the results of round
%     $i+1$.\end{theorem}\label{theorem:rounds}
%   % todo: put this somewhere
%   % \paragraph{Accountability:} In the context of Dissent, accountability refers
%   % to the ability of a protocol to detect and exclude participants who disrupt
%   % the protocol \cite{sec}, while proving that the disruptor did
%   % indeed disrupt the protocol. Such a mechanism is necessary in peer-to-peer
%   % protocols like Dining Cryptographers in which a single disruptor can make
%   % the result of a round unusable. In Dissent, accountability checks occur
%   % without revealing the link between any message and its sender --- moreover,
%   % it is not possible to deliberately exclude a participant on the basis of
%   % valid messages the participant sends. That is, the Dissent accountability
%   % mechanism does not break anonymity.
% \subsection{Dissent in Numbers}
% An instance of Dissent in Numbers consists of $\NumClients$ clients and
% $\NumServers$ servers.  Communication takes place in \emph{rounds}, wherein each
% client has an opportunity to broadcast a message to the entire client set.
% Assuming $\NumHonest$ of the $\NumClients$ clients are honest, each client is
% guaranteed that, at the protocol level, its message will be anonymous among the
% $\NumHonest$ honest clients unless all servers collude with each
% other.\toadd{Move the security properties to the goals section and only describe
% the functionality here} Since all clients receive each client's messages in a
% deterministic order, there is a well-defined sequence of rounds which we can
% associate with a monotonically increasing \emph{round ID}.
%
% \subsection{Dissent-in-Numbers Layer}
% \todo[inline]{TODO: Describe uses of DIN}
% \missingfigure{TODO: Draw the topology and a flow chart showing pseudonyms/public
% keys}
%
