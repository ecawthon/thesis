\toadd{Especially if I do take out the related work, I should add a short
intro to why Dissent is a good thing to use}
  Dissent is an alternative to Tor that provides provable anonymity even if
  only one server in the network is honest\cite{corrigan-gibbs_dissent:_2010}.
  \section{Protocol Explanation}
  In its present form, a Dissent cluster consists of $m$ servers and $n$
  connected clients\cite{wolinsky_dissent_2012}. Provable anonymity is
  achieved through a modified version of the Dining Cryptographers
  problem\cite{chaum_dining_1988}: each client $i$ shares a secret $K_{ij}$
  with each server $j$. Communication proceeds in rounds, within which each
  client has a designated $k$-bit slot.  Before any messages are sent, a
  secure shuffle\cite{neff_verifiable_2001} assigns each client to a slot so
  that the owner of a slot is the only node in the system which knows who owns
  that slot.  In any client $s$'s slot, every client and every server
  generates $k$ bits of random noise seeded with each of its shared secrets
  $K_{ij}$, and combines these with an exclusive or (xor) operation to produce
  that node's ciphertext. Client $s$ also combines (via xor) a $k$-bit message
  with its noise to create its ciphertext. The combination (via xor) of all
  clients' and servers' ciphertext includes the noise stream associated with
  each shared secret twice, and so all noise cancels out and client $s$'
  message is revealed. However, since deciphering this requires the
  participation of all nodes in the system, it is impossible to tell which
  client transmitted a message in a given slot. Dissent also incorporates an
  accountability mechanism, allowing any node that disrupts the protocol to be
  detected and removed from the cluster
  \cite{corrigan-gibbs_proactively_2013}.

  The original Dissent was fully peer-to-peer
  \cite{corrigan-gibbs_dissent:_2010}. The shift to a client-server model
  allows for significantly improved performance, but it introduces several new
  concerns, particularly relating to misbehaving servers, a new class of DoS
  attacks, and group formation.

  \section{Properties}
  \section{Evaluation}
