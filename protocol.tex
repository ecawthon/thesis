  % \item Further, intermediate vote counts are necessarily public. We prove
  %   generally in Appendix~\ref{Appendix:SecretProof} that we cannot do better.
% \end{itemize}
We operate a simple voting protocol based on linkable ring signatures
\cite{lrs}, on top of a Dissent instance that provides instigator anonymity,
along with acccountability to handle byzantine faults.
\section{Initial formation}
We assume the member set is well known, and that every member has a secure
channel through which it can communicate with every other member, and that
members remain connected. Initially, the
members organize themselves into a Peer-to-Peer Dissent
cluster \cite{p2pd} using some consensus protocol such
as Byzantine Paxos, as discussed in \cite{sec}.

The Dissent configuration file specifies the size and ordering of clients'
message slots\todogrunt{Need to define client and slot somewhere}. which will
consist of, for each of the $n$ clients: space for a \texttt{Ballot}, and space
for up to $n$ signatures so the client can sign whatever other proposals are in
play.  We only allow any given client to have one proposal at any given time.

\section{Peer-to-Peer Layer}
Every \texttt{Member} maintains a TCP connection to every other \texttt{Member},
and uses this to execute the peer to peer version of Dissent \cite{sec}. The
primary limitation of this version is performance --- there is a detailed
security analysis demonstrating the anonymity preserving properties of this
layer, and so we use it as a fallback layer for reconfiguring if servers
misbehave by dropping out, and for initial setup of the Dissent in Numbers (DIN)
layer.

For any instance of the DIN layer, there is a canonical \texttt{Manifest}
maintained at the P2P layer describing its parameters.  The \texttt{Manifest}
consists of
\begin{itemize}
  \item A \texttt{Roster} $R$, mapping public keys to IP addresses for all
    \texttt{Client}s,
  \item A \texttt{Servers} list $S$, which is a subset of $R$,
  \item A ratio $t$ specifying what proportion of \texttt{Member}s must
    agree to a change in the composition of $R$ or $S$ in order for the change
    to take effect
\end{itemize}

When a vote to change the \texttt{Manifest} passes, \toadd{Describe how to
  start a Dissent instance. This is a solved problem, I just need to summarize
it.}

\section{Dissent-in-Numbers Layer}
The communication involved in establishing the \texttt{Manifest} takes place
over an instance of Dissent in Numbers \cite{din}. We sketch a
black box model of Dissent in Numbers as it relates to our protocol.

An instance of DIN consists of $n$ clients and $m$ servers. Communication takes
place in \emph{rounds}, wherein each client has an opportunity to broadcast a
message to the entire client set. Assuming $k$ of the $n$ clients are honest,
each client is guaranteed that, at the protocol level, its message will be
anonymous among the $k$ honest clients unless all servers collude with each
other.\todoword{Move the security properties to the goals section and only
describe the functionality here} Since all clients receive each client's
messages in a deterministic order, there is a well-defined sequence of rounds
which we can associate with a monotonically increasing \emph{round ID}.

\missingfigure{Illustrate the round message format, which will consist of, for
  each of the $n$ clients: space for a \texttt{Ballot}, and space for up to
  $n$ signatures so the client can sign whatever other proposals are in play.
We only allow any given client to have one proposal at any given time.}

Within a round, each \texttt{Member} may \todosubst{Jamming by proposing
ballots??? Infinite timeouts???} transmit a \texttt{Petition}. A
\texttt{Petition} consists of:
\begin{itemize}
  \item A proposed \emph{Manifest}, as described above,
  \item A \emph{Link Scope}\todogrunt{Explain}
  % \item A collection of \texttt{Signatures}\todogrunt{Explain}
  \item A \emph{Round ID} when the ballot will expire.
\end{itemize}

Once a \emph{Ballot} has been proposed, the other \texttt{Member}s have the
opportunity to \emph{vote}. A \texttt{Member} votes by transmitting the most
recent version of the \texttt{Ballot}, but with the \texttt{Signatures} field
modified to include the proposed \texttt{Manifest} signed with the voting
\texttt{Member}'s private key for this link scope.\todoword{I think this is
wrong}

By the designated expiration round, all \texttt{Member}s have enough
information to determine whether or not the \texttt{Ballot} \emph{passes}:
Each \texttt{Member} should verify all signatures on the most recent
version\todosubst{What if there are conflicting versions?}
and compare the total number of valid signatures to the threshold $t$. If the
\texttt{Ballot} passes, the new server set should immediately set up the
next iteration of the DIN layer.\toadd{finish describing new setup}

\subsection{Limitations and Non-Goals}
  \paragraph{Intersection Attacks:} The \texttt{Peer} and \texttt{Member} sets are known. In Dissent in
    Numbers, clients need not know the IP addresses of any other clients. We
    believe it is useful to have a protocol for a group with static membership.
    If there is membership churn, our protocol remains vulnerable to
    intersection attacks.
  \paragraph{Coercion-Resistance:} In order to use anonymous ring signatures for voting, it is necessary
    for each signature within a scope to correspond to a specific member. An
    adversary with the power to coerce \texttt{Member}s into revealing their
    private keys after the fact may prove that a particular member voted a
    particular way\cite{lrs}.
    \todogrunt{Need some kind of analysis of why this is still
    useful despite that}
  % todo: put this somewhere
  % \paragraph{Accountability:} In the context of Dissent, accountability refers
  % to the ability of a protocol to detect and exclude participants who disrupt
  % the protocol \cite{sec}, while proving that the disruptor did
  % indeed disrupt the protocol. Such a mechanism is necessary in peer-to-peer
  % protocols like Dining Cryptographers in which a single disruptor can make
  % the result of a round unusable. In Dissent, accountability checks occur
  % without revealing the link between any message and its sender --- moreover,
  % it is not possible to deliberately exclude a participant on the basis of
  % valid messages the participant sends. That is, the Dissent accountability
  % mechanism does not break anonymity.
  \paragraph{Forward Progress:} A protocol that guarantees forward progress
  given certain conditions will eventually make progress as long as those
  conditions are met. More strict bounds on what ``eventually'' means are
  possible. In the original Dissent, for example, forward progress can be
  guaranteed if all clients follow the protocol and remain online, but not
  otherwise: The accountability mechanism was so arduous that $f$ disrupting
  clients could prevent any messages from being transmitted for $f$ hours
  \cite{verdict}. Protocols that make use of quorums
  rather than being fully peer-to-peer are able to provide stronger guarantees
  of forward progress \cite{paxos}. Since we do not handle traditional fault
  tolerance or network churn,  this is an area
  for future work \todo{lulz}
  \paragraph{Active attacks:}
  Global traffic analysis attacks only require the adversary to be able to
  monitor global traffic. If the adversary can also modify or generate
  traffic, several other attacks are possible. The adversary can launch
  \emph{man-in-the-middle} (MITM) attacks in which it impersonates Alicia,
  Badru, or one or more Tor nodes. The adversary can launch \emph{Sybil}
  attacks, in which many different Tor clients controlled by the same
  adversary join the network as individual clients. Either of these can be
  used to create \emph{Denial-of-Service} (DoS) attacks, which might either
  prevent users from connecting to Tor at all, or force Tor traffic to go
  through particular, potentially adversary-controlled, Tor nodes ---
  de-anonymizing the users.
