\section{Peer-to-Peer Layer}
Every \texttt{Member} maintains a TCP connection to every other \texttt{Member}.

For any instance of the DIN layer, there is a canonical \texttt{Manifest}
maintained at the P2P layer describing its parameters.  The \texttt{Manifest}
consists of
\begin{itemize}
  \item A \texttt{Roster} $R$, mapping public keys to IP addresses for all
    \texttt{Client}s,
  \item A \texttt{Servers} list $S$, which is a subset of $R$,
  \item A ratio $t$ specifying what proportion of \texttt{Member}s must
    agree to a change in the composition of $R$ or $S$ in order for the change
    to take effect
\end{itemize}

When a vote to change the \texttt{Manifest} passes, \toadd{Describe how to
  start a Dissent instance. This is a solved problem, I just need to summarize
it.}

\section{Dissent-in-Numbers Layer}
The communication involved in establishing the \texttt{Manifest} takes place
over an instance of Dissent in Numbers \cite{wolinsky_dissent_2012}. We sketch a
black box model of Dissent in Numbers as it relates to our protocol.

An instance of DIN consists of $n$ clients and $m$ servers. Communication takes
place in \emph{rounds}, wherein each client has an opportunity to broadcast a
message to the entire client set. Assuming $k$ of the $n$ clients are honest,
each client is guaranteed that, at the protocol level, its message will be
anonymous among the $k$ honest clients unless all servers collude with each
other.\todoword{Move the security properties to the goals section and only
describe the functionality here} Since all clients receive each client's
messages in a deterministic order, there is a well-defined sequence of rounds
which we can associate with a monotonically increasing \emph{round ID}.

The DIN configuration file specifies the size and ordering of clients' message
slots\todogrunt{Need to define client and slot somewhere}.

\missingfigure{Illustrate the round message format, which will consist of, for
  each of the $n$ clients: space for a \texttt{Ballot}, and space for up to
  $n$ signatures so the client can sign whatever other proposals are in play.
We only allow any given client to have one proposal at any given time.}

Within a round, each \texttt{Member} may \todosubst{Jamming by proposing
ballots??? Infinite timeouts???} transmit a \texttt{Ballot}. A \texttt{Ballot}
consists of:
\begin{itemize}
  \item A proposed \emph{Manifest}, as described above,
  \item A \emph{Link Scope}\todogrunt{Explain}
  \item A collection of \texttt{Signatures}\todogrunt{Explain}
  \item A \emph{Round ID} when the ballot will expire.
\end{itemize}

Once a \emph{Ballot} has been proposed, the other \texttt{Member}s have the
opportunity to \emph{vote}. A \texttt{Member} votes by transmitting the most
recent version of the \texttt{Ballot}, but with the \texttt{Signatures} field
modified to include the proposed \texttt{Manifest} signed with the voting
\texttt{Member}'s private key for this link scope.\todoword{I think this is
wrong}

By the designated expiration round, all \texttt{Member}s have enough
information to determine whether or not the \texttt{Ballot} \emph{passes}:
Each \texttt{Member} should verify all signatures on the most recent
version\todosubst{What if there are conflicting versions?}
and compare the total number of valid signatures to the threshold $t$. If the
\texttt{Ballot} passes, the new server set should immediately prepare for the
next iteration of the DIN layer.\toadd{finish describing new setup}

