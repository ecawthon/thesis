\section{Peer-to-Peer Layer}
Every \texttt{Member} maintains a TCP connection to every other \texttt{Member}.
\todo[inline]{TODO: set up the notion of a state}
At any logial time slice, there is a canonical \emph{Manifest} describing the
state of the DIN layer. The \emph{Manifest} consists of
\begin{itemize}
  \item A
  \item A ratio \emph{t}
\end{itemize}

% Group Policy
% ---------------------
%  `Member Set`: A `Member Set` _M = (t, C, S)_ defines the parameters of a
%  Dissent cluster's membership and policy. It consists of a ratio _t_ specifying
%  what proportion of members must agree to a change in the composition of _M_ for
%  the change to be implemented; and two possibly overlapping sets of `Peer`s: The
%  client set _C_ and the server set _S_.
%
%  `Roster`: A `Roster` _R = (M, L, S)_ consists of a `Member Set` _M_, a
%  `linkScope` _L_, and a collection of signatures _S_.
%
% Peer operations
% ================================================================================
% Joining
% --------------------------------------------------------------------------------
% A `Peer` may start its own cluster consisting of itself at any time.
%
% To join an existing cluster, a `Peer` _P_ must know the address of at least one
% member _M_.
% 1. _P_ sends a `Join_request` message to one or more members
% 2. _M_ initiates a vote on whether _P_ can join
% Member operations
% --------------------------------------------------------------------------------
% Unless otherwise specified, all `Member` messages are broadcast as regular
% Dissent messages.
%
% In its own slot, any `Member` may:
% - `Propose` a new `Roster` by specifying a proposed
%     `Member Set`, and an arbitrary `linkScope` specific to this `Roster`, and
%     signing it according to the `linkScope`. It also includes a round number for
%     when the proposal should take effect.
% - `Vote` by adding its signature to an existing `Roster` _(M, L, S_) by
%    broadcasting _(M, L, S')_.
%
% Safety Analysis
% ================================================================================
% (What does it mean for a new roster to take effect?)
% A new `Roster` is enacted by launching a new configuration.
% (Do we need to re-shuffle or just reset the round?)
% (We need a base case.)
% Once any server learns that it is a server, it broadcasts that it is ready for
% clients.
% Clients can also enqueue themselves (?)
% (what's a broadcast?)
%
% Security Analysis
% ================================================================================
% Limitations
% -----------
% * All clients know all other clients' IP addresses. This is not true in DiN.
%
% TODO
% =============
% * Distinguishing between members who are actually participating and members who
%     are allowed to participate - this should not require all members to
%     participate every round.
% * How to verify the threshold has been met - need to decide if it's threshold of
%     members or threshold of active participants
% * Member sets should probably have IDs - it should be legal for a Peer to be a
%     member of multiple clusters. Further, this allows for distinguishing between
%     forming a new set and modifying an existing one (wouldn't want to be able to
%     propose a tiny member set and then say the threshold of that new tiny member
%     set is met so the new set takes effect)
% * How role transitions actually occur
%
% Possible additions
% ==================
% * Adopting the intersection of multiple rosters with different signature sets?
% * Proposing could involve having garbage for all the signature slots - once
%     enough signature slots are readible, done.
% * Idea: Don't deal with churn.
% \subsection{Peer-to-Peer Layer}
