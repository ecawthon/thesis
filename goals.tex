In transforming Dissent in Numbers\cite{wolinsky_dissent_2012} to a fully
decentralized context, some differential terminology is necessary. We define
several components of our protocol before moving on to discuss desired security
properties.

\section{Terminology}
The protocol consists of two logical layers: A Dissent in Numbers layer, where
most communication takes place, and an underlying peer-to-peer layer where
management of the Dissent configuration occurs. We refer to the entire protocol
as Dissent, and refer to the Dissent in Numbers layer as \textit{DIN}.

We refer to any single running instance of our protocol on a single machine as a
\texttt{Peer}. When a \texttt{Peer} is part of a Dissent cluster, it is a
\texttt{Member} of that cluster.

Within the DIN layer, each \texttt{Member} takes on the role of a
\texttt{Client} as described in \cite{wolinsky_dissent_2012}. In addition, a
collectively-chosen subset of these \texttt{Member}s also runs a \texttt{Server}
process (as described in \cite{wolinsky_dissent_2012}).

\toadd{Probably not relevant to mention Low
Latency Dissent, but may be worth reworking this paragraph to refer to the
possibility of generalizing}
Based on membership operations, \texttt{Member}s may take on additional roles.
In Dissent in Numbers a subset of \texttt{Member}s would also be
\texttt{Server}s. In Low-Latency Dissent one \texttt{Member}s would
also be (a) \texttt{Relay}(s). These roles are subject to change with future
membership proposals.

\section{Assumptions/Non-Goals}
\toadd{turn bullet points into paragraphs}
\toadd{formalize in distributed systems verification terms
(initially, next, etc.)}
We present the assumptions and goals of our model in informal terms here,
before formalizing the definitions as part of proving the properties are
satisfied in Chapter~\ref{Chapter:Proofs}.
\subsection{Assumptions}
\begin{itemize}
  \item There is an initial \texttt{Manifest} describing a Dissent-in-Numbers
    setup, known to all \texttt{Members} it describes.
  \item While the \texttt{Member} set may change according to votes, the
    \texttt{Peer} set is fixed.
\end{itemize}
\subsection{Limitations}
\begin{itemize}
  \item The \texttt{Peer} and \texttt{Member} sets are known. In Dissent in
    Numbers, clients need not know the IP addresses of any other clients.
  \item In order to use anonymous ring signatures for voting, it is necessary
    for each signature within a scope to correspond to a specific member. An
    adversary with the power to coerce \texttt{Member}s into revealing their
    private keys after the fact may prove that a particular member voted a
    particular way.\todogrunt{Need some kind of analysis of why this is still
    useful despite that}
  \item Further, intermediate vote counts are necessarily public. We prove
    generally in Appendix~\ref{Appendix:SecretProof} that we cannot do better.
\end{itemize}

\section{Goals}
  \subsection{Potential Goals}
  \toadd{Taken from lit review, need to remove redundancy}
    \paragraph{Verifiability:} A protocol is verifiable if its output can be
    inspected to confirm that the protocol was carried out correctly. A simple
    example of this is signing a message with the private key associated with a
    well-known public key: Anyone who knows the public key can verify the
    validity of the signature. Dissent \cite{corrigan-gibbs_proactively_2013}
    and the Neff shuffle \cite{neff_verifiable_2001} both use zero-knowledge
    proofs to achieve verifiability.
    \paragraph{Accountability:} In the context of Dissent, accountability refers
    to the ability of a protocol to detect and exclude participants who disrupt
    the protocol \cite{syta_security_2014}, while proving that the disruptor did
    indeed disrupt the protocol. Such a mechanism is necessary in peer-to-peer
    protocols like Dining Cryptographers in which a single disruptor can make
    the result of a round unusable. In Dissent, accountability checks occur
    without revealing the link between any message and its sender --- moreover,
    it is not possible to deliberately exclude a participant on the basis of
    valid messages the participant sends. That is, the Dissent accountability
    mechanism does not break anonymity.
    \paragraph{Forward Progress:} A protocol that guarantees forward progress
    given certain conditions will eventually make progress as long as those
    conditions are met. More strict bounds on what ``eventually'' means are
    possible. In the original Dissent, for example, forward progress can be
    guaranteed if all clients follow the protocol and remain online, but not
    otherwise: The accountability mechanism was so arduous that $f$ disrupting
    clients could prevent any messages from being transmitted for $f$ hours
    \cite{corrigan-gibbs_proactively_2013}. Protocols that make use of quorums
    rather than being fully peer-to-peer are able to provide stronger guarantees
    of forward progress \cite{lamport_part-time_1998}.
    \paragraph{Anonymity:} A protocol guarantees anonymity in some operation a
    client can complete if the output of that operation is unlinkable (or, more
    precisely, cryptographically very difficult to link) to the client who
    completed it \cite{corrigan-gibbs_dissent:_2010}. Dissent makes use of
    pseudonyms to provide this, separating the protocol correctness and
    accountability layer from the layer in which messages are revealed. Group
    membership voting could conceivably take place at either.

\subsection{Desired Properties}
\paragraph{Progress:}
\begin{itemize}
  \item If a valid \texttt{Member} wants to propose a \texttt{Ballot}, it can
    do so within a finite number of rounds
  \item If a \texttt{Ballot} is proposed, every \texttt{Member} should have
    the opportunity to vote on it
  \item If $t$ of $n$ clients vote for a \texttt{Ballot}, that \texttt{Ballot}
    should take effect within a finite number of rounds.
\end{itemize}
\paragraph{Decentralization:}
\begin{itemize}
  \item No changes to the manifest occur without $t$ of the $n$ clients'
    approval of the new manifest.
  \item From any possible state, a cadre of $t$ of $n$ clients can cause the
    manifest (and corresponding configuration) to change to any other valid
    manifest.\todoword{Define state better}
\end{itemize}
\paragraph{Security:}
\begin{itemize}
  \item \texttt{Member}s who disrupt the protocol can be detected and removed,
    without deanonymizing participants\todo[color=yellow]{TODO: Cite
      \cite{corrigan-gibbs_proactively_2013}; this is solved}
  \item  Votes should be
  \begin{itemize}
    \item Anonymous --- No member can learn which member voted which way.
    \item Verifiable --- Once a vote is complete, any member can, given
      a canonical \texttt{Ballot} containing $s$ signatures, verify how many
      of the $s$ correspond to distinct, valid members' votes, and also that
      their own vote is included.
  \end{itemize}
  \item While there are at least $t$ honest \texttt{Member}s, no changes to the
    topology should occur without $t$ of the $n$ clients' approval
  \item If there are fewer than $t$ honest \texttt{Member}s, those
    \texttt{Member}s should still retain strong anonymity among the honest
    clients, even if they no longer control the topology\todogrunt{This is
    fuzzy, explain better}
\end{itemize}
\subsection{Adversary Model}
We assume the adversary can
\begin{itemize}
  \item Monitor all network traffic
  \item Control some fraction of the \texttt{Peers}, which may send arbitrary
    messages to each other through secret channels and to the group through
    ordinary channels
  \item Inject traffic\todosubst{Do we? Specify this}
\end{itemize}
