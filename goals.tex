In transforming Dissent in Numbers\cite{wolinsky_dissent_2012} to a fully
decentralized context, some differential terminology is necessary. We define
several components of our protocol before moving on to discuss desired security
properties.

\section{Terminology}
The protocol consists of two logical layers: A Dissent in Numbers layer, where
most communication takes place, and an underlying peer-to-peer layer where
management of the Dissent configuration occurs. We refer to the entire protocol
as Dissent, and refer to the Dissent in Numbers layer as \textit{DIN}.

We refer to any single running instance of our protocol on a single machine as a
\texttt{Peer}. When a \texttt{Peer} is part of a Dissent cluster, it is a
\texttt{Member} of that cluster.

Based on membership operations, \texttt{Member}s may take on additional roles.
In Dissent in Numbers a subset of \texttt{Member}s would also be
\texttt{Server}s. In Low-Latency Dissent one \texttt{Member}s would
also be (a) \texttt{Relay}(s). These roles are subject to change with future
membership proposals.\todo{TODO: Probably not relevant to mention Low Latency
Dissent, but may be worth reworking this paragraph to refer to the possibility
of generalizing}

