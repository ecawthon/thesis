\section{Goals}
\subsection{Verifiability:}
We provide individual and universal verifiability of votes, as defined in
Section~\ref{Subsection:evoting}. More precisely, we provide the following:
\paragraph{Authentication:} At the end of an election, all members either know
that all votes were cast by group members, or they know which ones were not
valid.\todogrunt{and can kick out, potentially? For another time\ldots}
\paragraph{Individual Verifiability:} Each member can verify that its vote has
been included in the canonical result\todosubst{double check how DIN handles
equivocation --- i.e. that it can be detected and not just that it doesn't
de-anonymize}
\paragraph{Universal Verifiability:} At the end of an election, all members
either know the result of the vote, or know that the result is
invalid\todoword{formalize}
\subsection{Anonymity:}
We provide anonymity in two important domains:
\paragraph{Ballot Secrecy:} The adversary should gain no information about which
member cast which vote in a given election.\todogrunt{Formalize/define
``member'', ``vote'', and ``election''}
\paragraph{Instigator Secrecy:} The adversary should gain no information about
which member proposed any given proposal.\todogrunt{flesh out}
\todo[caption={},color=orange,inline]{
  \begin{spacing}{1.0}TODO: There is apparent redundancy because
  DIN already has shuffled slots, so why would we need LRS instead of just
  plaintext writing the vote? I should write/think through at least one of:
  \begin{itemize}
  \item A special case for before DIN is set up where the LRS anonymity is
    necessary, and/or
  \item Efficiency advantage to not having to re-do the Neff shuffle every time
    there's an election --- would we have to reshuffle anyway to switch servers?
    (potentially not!)
\end{itemize}\end{spacing}}
\subsection{Reversibility:}

% In transforming Dissent in Numbers\cite{wolinsky_dissent_2012} to a fully
% decentralized context, some differential terminology is necessary. We define
% several components of our protocol before moving on to discuss desired security
% properties.
%
% \section{Terminology}
% The protocol consists of two logical layers: A Dissent in Numbers layer, where
% most communication takes place, and an underlying peer-to-peer layer where
% management of the Dissent configuration occurs. We refer to the entire protocol
% as Dissent, and refer to the Dissent in Numbers layer as \textit{DIN}.
%
% We refer to any single running instance of our protocol on a single machine as a
% \texttt{Peer}. When a \texttt{Peer} is part of a Dissent cluster, it is a
% \texttt{Member} of that cluster.
%
% Within the DIN layer, each \texttt{Member} takes on the role of a
% \texttt{Client} as described in \cite{wolinsky_dissent_2012}. In addition, a
% collectively-chosen subset of these \texttt{Member}s also runs a \texttt{Server}
% process (as described in \cite{wolinsky_dissent_2012}).
%
% \toadd{Probably not relevant to mention Low
% Latency Dissent, but may be worth reworking this paragraph to refer to the
% possibility of generalizing}
% Based on membership operations, \texttt{Member}s may take on additional roles.
% In Dissent in Numbers a subset of \texttt{Member}s would also be
% \texttt{Server}s. In Low-Latency Dissent one \texttt{Member}s would
% also be (a) \texttt{Relay}(s). These roles are subject to change with future
% membership proposals.
%
%
% \section{Goals}
% \subsection{Desired Properties}
% \paragraph{Progress:}
% \begin{itemize}
%   \item If a valid \texttt{Member} wants to propose a \texttt{Ballot}, it can
%     do so within a finite number of rounds
%   \item If a \texttt{Ballot} is proposed, every \texttt{Member} should have
%     the opportunity to vote on it
%   \item If $t$ of $n$ clients vote for a \texttt{Ballot}, that \texttt{Ballot}
%     should take effect within a finite number of rounds.
% \end{itemize}
% \paragraph{Decentralization:}
% \begin{itemize}
%   \item No changes to the manifest occur without $t$ of the $n$ clients'
%     approval of the new manifest.
%   \item From any possible state, a cadre of $t$ of $n$ clients can cause the
%     manifest (and corresponding configuration) to change to any other valid
%     manifest.\todoword{Define state better}
% \end{itemize}
% \paragraph{Security:}
% \begin{itemize}
%   \item \texttt{Member}s who disrupt the protocol can be detected and removed,
%     without deanonymizing participants\todo[color=yellow]{TODO: Cite
%       \cite{corrigan-gibbs_proactively_2013}; this is solved}
%   \item  Votes should be
%   \begin{itemize}
%     \item Anonymous --- No member can learn which member voted which way.
%     \item Verifiable --- Once a vote is complete, any member can, given
%       a canonical \texttt{Ballot} containing $s$ signatures, verify how many
%       of the $s$ correspond to distinct, valid members' votes, and also that
%       their own vote is included.
%   \end{itemize}
%   \item While there are at least $t$ honest \texttt{Member}s, no changes to the
%     topology should occur without $t$ of the $n$ clients' approval
%   \item If there are fewer than $t$ honest \texttt{Member}s, those
%     \texttt{Member}s should still retain strong anonymity among the honest
%     clients, even if they no longer control the topology\todogrunt{This is
%     fuzzy, explain better}
% \end{itemize}
% \subsection{Adversary Model}
% We assume the adversary can
% \begin{itemize}
%   \item Monitor all network traffic
%   \item Control some fraction of the \texttt{Peers}, which may send arbitrary
%     messages to each other through secret channels and to the group through
%     ordinary channels
%   \item Inject traffic\todosubst{Do we? Specify this}
% \end{itemize}
\section{Assumptions/Non-Goals}
\toadd{formalize in distributed systems verification terms
(initially, next, etc.)}
We present the assumptions and goals of our model in informal terms here,
before formalizing the definitions as part of proving the properties are
satisfied in Chapter~\ref{Chapter:Proofs}.
\subsection{Assumptions}
\begin{itemize}
  \item There is an initial \texttt{Manifest} describing a Dissent-in-Numbers
    setup, known to all \texttt{Members} it describes.
  \item While the \texttt{Member} set may change according to votes, the
    \texttt{Peer} set is fixed.
\end{itemize}
\subsection{Limitations}
\begin{itemize}
  \item The \texttt{Peer} and \texttt{Member} sets are known. In Dissent in
    Numbers, clients need not know the IP addresses of any other clients.
  \item In order to use anonymous ring signatures for voting, it is necessary
    for each signature within a scope to correspond to a specific member. An
    adversary with the power to coerce \texttt{Member}s into revealing their
    private keys after the fact may prove that a particular member voted a
    particular way.\todogrunt{Need some kind of analysis of why this is still
    useful despite that}
  \item Further, intermediate vote counts are necessarily public. We prove
    generally in Appendix~\ref{Appendix:SecretProof} that we cannot do better.
\end{itemize}
