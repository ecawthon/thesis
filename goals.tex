We identify three crucial properties for a group decisionmaking protocol that
keeps power decentralized.

First, the protocol's computation must be distributed. If the adversary has
control of the network infrastructure, it can easily block access to any central
component of a protocol (such as a Tor relay \tocite or voting
registrar\todoword{use the literature terminology}). If blocking one or a small
subset of nodes in the group is sufficient to break the protocol, this again
weakens our decentralization guarantees and renders the protocol less useful. A
fault tolerant, distributed protocol, however, can easily recover from arbitrary
changes in the network topology.

Next, the protocol must be verifiable --- any particiapnt or external auditor
should be able to examine a transcript of the protocol and determine whether or
not it was performed honestly. Without this property, it would be necessary to
trust that some portion of the group acted honestly without verification ---
this constitutes centralized trust, and is therefore inadequate.

Finally, the protocol must be anonymous --- it should be computationally hard to
attribute any vote or any petition to any specific participant. This is
essential to preserve the safety of members suggesting or voting for unpopular
proposals.

We now formalize these notions:
\section{Distributed Computation}\label{Subsection:distr}
\toadd{Some questionable notion of this that we meet}
\section{Verifiability}\label{Section:verif}
A protocol is verifiable if its output can be inspected to confirm that the
protocol was carried out correctly. A simple example of this is signing a
message with the private key associated with a well-known public key: Anyone
who knows the public key can verify the validity of the signature.

Voting protocols can be evaluated in their provision of three different
kinds of verifiability \cite{kremer_election_2010}: \emph{individual}
verifiability ensures that a voter can verify their vote was included
correctly. \emph{Universal} verifiability requires that anybody can verify
the election result correctly represents the collection of ballots cast.
Finally, \emph{Eligibility} verifiability allows anybody to verify that
only eligible voters voted, and that each voter voted only once.

  \subsection{Individual}
  A group management protocol provides \emph{individual verifiability} if, in
  any Election State $(\State, \VarPetition, \SetVotes)$, any member
  $\VarMember$ either knows its own vote is correctly represented in $\SetVotes$
  (that is, either $\VarMember$ voted and $\VarMember$'s signature for
  $\VarPetition.\VarLinkScope$ \todo{TODO: more params; define lrs} is contained
  in $\SetVotes$, or $\VarMember$ did not vote and $\VarMember$'s signature is
  absent from $\SetVotes$), or can produce a zero-knowledge proof that the
  Election State is invalid.\todo{TODO: define invalid}

  \subsection{Universal}
  A group management protocol provides \emph{universal verifiability} if, in any
  finished election state, $(\State, \VarPetition, \SetVotes)$ , anybody can
  verify that $\SetVotes$ is a valid signing of $\VarPetition$ or else produce a
  proof that it is not. Consequently, any auditor (member or otherwise) can
  verify the canonical value of $\VarManifest.\FunEval(\ElectionState)$.

  \subsection{Eligibility}
  A group management protocol provides \emph{eligibility verifiability} if, in
  any finished election state, anybody can verify that all elements of
  $\SetVotes$ were cast by \KwMember s of the current cluster.

\section{Anonymity}
Members of any group often face reprocussions if they participate in group
governance in ways that run contrary to the interests of other members of
the group. For this reason, election protocols often incorporate some notion
of anonymity. A protocol guarantees \emph{anonymity} in some operation a
client can complete if the output of that operation is unlinkable (or, more
precisely, cryptographically very difficult to link) to the participant who
completed it\cite{ford_hiding_2014}.

We are interested in two types of anonymity: First, within an election,
each voter's confidentiality should be preserved. Second, the instigator of an
election, who may also be the author of the proposed petition, should be
anonymous.

  \subsection{For Instigators}
  A group management protocol provides \emph{instigator anonymity} if, during
  and after any election, no member and no outside observer can determine
  which member proposed the ballot in question.

  \subsection{Of Ballots}
  A group management protocol provides \emph{secret ballots} if, during and
  after any election, either no outside observer can reconstruct which member
  submitted which vote, or no outside observer can reconstruct how any
  member voted. The same restrictions apply to knowledge gained by other
  participants, except that each member can trivially reconstruct its own vote.

