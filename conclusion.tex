We have argued that verifiability is not enough for a
post-Snowden electronic voting protocol. We have specified and analyzed the
properties a voting protocol would need to provide in order to conform to a
revised trust model, wherein peers are not required to trust one another, and
wherein the adversary is assumed to be able to monitor all network traffic. We
have further contributed the first voting protocol we know of that provides the
verifiability guarantees of the electronic voting literature, the strong
anonymity of Dissent, and the fully decentralized trust model of Byzantine
peer-to-peer systems.

Future work will involve a more detailed specification and analysis of the
protocol sketched in Chapter~\ref{Chapter:Protocol}. Work has already begun on
implementing it and integrating the election of servers into future versions of
Dissent, utilizing the implementation of linkable ring signatures available in
Go \cite{golrs}. Once it is known to be secure, the changes should be fully
implemented and performance should be analyzed in the face of various kinds of
disruption, and also client churn.

As networked communication becomes increasingly integral to the communications
of humans in collectives, questions of security too become essential to analysis
of voting systems.  We believe this will be useful in constructing systems for
use by activists and governments alike as these infrastructures develop.

