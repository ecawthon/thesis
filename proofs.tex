\section{Terminology}
An \emph{election} encompasses the operation of the protocol between when a
ballot $B = (L, M rid)$ is first proposed and the round $rid$ --- that is, the
portion of the protocol where members are aware that that $B$ is being
considered, but in which no member knows the \emph{outcome} of the election.

A \emph{result} is a tuple $(M, G, rid)$ defining a manifest $M$ and a set of
signatures $G$.

A \emph{Manifest} consists of
\begin{itemize}
  \item A \texttt{Roster} $R$, mapping public keys to IP addresses for all
    \texttt{Client}s,
  \item A \texttt{Servers} list $S$, which is a subset of $R$,
  \item A function $t : G \to \{\textsc{True, False}\}$ mapping a set of
    signatures to an election result. So, if $t(g) = \textsc{True}$ for some
    outcome $g$\todo{clarify $G$}, then the proposal corresponding to $g$ should
    be adopted at the specified expiration round. A plausible example is the
    function which specifies what proportion of \texttt{Member}s must agree to a
    change in the composition of $R$ or $S$ in order for the change to take
    effect
\end{itemize}

An \emph{instigator} is a member who initiates an election.

A \emph{Ballot} is a tuple $(L, M, M', G)$, where $L$ is a unique bytestring
selected arbitrarily by the instigator, $M$ is the current manifest of the
cluster, $M'$ is a proposed new manifest, and $G$ is a collection of linkable
ring signatures from members of $M$ according to link scope $L$.


\section{Formal Properties and Correctness Arguments}
\toadd{clean up embedded results defs etc.}
  \subsection{Verifiability}
    \subsubsection{Individual}
    A group management protocol provides \emph{individual verifiability} if, in
    any result $r = (M, B, G, rid)$, any member $u$ who voted in the election
    either knows its own signature is included in $G$, or can produce a
    zero-knowledge proof that $r$ is invalid.

    Our protocol provides individual verifiability --- this follows from the
    properties of Dissent in Numbers and of linkable rigng signatures.%: Since every
    \subsubsection{Universal}
    A group management protocol provides \emph{universal verifiability} if, in
    any result $r = (M, B, G, rid)$, anybody can verify that $G$ is a valid
    signing of $B$ or else produce a proof that it is not. Consequently, any
    auditor (member or otherwise) can verify the canonical value of $M.t(G)$.

    Our protocol provides universal verifiability.%:
  \subsection{Anonymity}
  \todo[color=orange,inline]{TODO: formalize in terms of games}
    \subsubsection{For Instigators}
    A group management protocol provides \emph{instigator anonymity} if, during
    and after any election, no member and no outside observer can determine
    which member proposed the ballot in question.

    Our protocol provides this through Dissent in Numbers.
    \subsubsection{Of Ballots}
    A group management protocol provides \emph{secret ballots} if, during and
    after any election, either no outside observer can reconstruct which member
    submitted which vote, or no outside observer can reconstruct how any
    member voted. The same restrictions apply to knowledge gained by other
    participants, except that each member can trivially reconstruct its own vote.

    Our protocol provides this --- it follows directly from the properties of
    linkable ring signatures.
\subsection{Performance Notes}
Any changes to the arrangement of servers in Dissent in Numbers requires an
expensive, serial shuffle. Our protocol provides a way to change the topology
without having to redo the shuffle, so long as the client set remains the same.

This allows us to retain many of the stronger security properties of Hardened
Dissent\cite{sec} while also achieving the performance benefits
of Dissent in Numbers and Verdict in typical usage.
