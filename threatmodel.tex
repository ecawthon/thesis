Newly available information about vulnerabilities and global-scale
surveillance in today's centralized internet infrastructure has brought new
security considerations and threat models to the foreground of networked
system research. It has rendered a swath of anonymity and voting tools
obsolete, and poses a significant threat to those that remain. Group
management protocols aimed at dissidents must take this into account.

\section{Global Passive Adversary}
  To provide anonymity from an adversary like the N.S.A., a modern anonymity
  protocol must protect against several forms of attacks. Feigenbaum et.
  al.\cite{feigenbaum_seeking_2013} highlight several specific attacks to
  which onion routing is vulnerable:
  \paragraph{Global traffic analysis:}
  If the adversary can monitor most of the traffic on the internet globally,
  the adversary can with high probability see the link from Alicia to the Tor
  network and from the Tor exit relay to Badru. This means the adversary can
  observe that the messages Alicia sends correspond to messages Badru receives
  some short amount of time later. Even if the messages themselves are
  encrypted, the adversary can analyze the lengths and other metadata about
  the messages to correlate this traffic.
  \paragraph{Intersection attacks:}
  In general, it is possible to tell when users are using an anonymity service
  --- the anonymity comes from the difficulty of linking any particular user
  to particular messages produced by the service. Over time, however, the
  client set of an anonymity service is unlikely to remain fixed. A passive
  adversary monitoring the outputs of an anonymity service as well as the set
  of users connected can narrow the set of users who potentially, for example,
  updated a particular blog with a static pseudonym, by excluding users  not
  online during all updates to the blog.

\section{Limitations and Non-Goals}
  \paragraph{Intersection Attacks:} The \texttt{Peer} and \texttt{Member} sets
  are known. In Dissent in Numbers, clients need not know the IP addresses of
  any other clients. We believe it is useful to have a protocol for a group with
  static membership.  If there is membership churn, our protocol remains
  vulnerable to intersection attacks.
  \paragraph{Coercion-Resistance:} In order to use anonymous ring signatures for
  voting, it is necessary for each signature within a scope to correspond to a
  specific member. An adversary with the power to coerce \texttt{Member}s into
  revealing their private keys after the fact may prove that a particular member
  voted a particular way\cite{lrs}.lrs

``An important and fundamental downside,
however, is that linkable signatures do NOT offer forward-secure anonymity. If
an anonymity set member's private key is later released, it is trivial to check
whether or not that member produced a given signature. Also, anonymity set
members who did NOT sign a message could (voluntarily or under coercion) prove
that they did not sign it, e.g., simply by signing some other message in that
linkage context and noting that the resulting linkage tag comes out different.
Thus, linkable anonymous signatures are not appropriate to use in situations
where there may be significant risk that members' private keys may later be
compromised, or that members may be persuaded or coerced into revealing whether
or not they produced a signature of interest.''
\cite{golrs}
    \todogrunt{Need some kind of analysis of why this is still
    useful despite that}
  \paragraph{Forward Progress:} A protocol that guarantees forward progress
  given certain conditions will eventually make progress as long as those
  conditions are met. More strict bounds on what ``eventually'' means are
  possible. In the original Dissent, for example, forward progress can be
  guaranteed if all clients follow the protocol and remain online, but not
  otherwise: The accountability mechanism was so arduous that $f$ disrupting
  clients could prevent any messages from being transmitted for $f$ hours
  \cite{verdict}. Protocols that make use of quorums
  rather than being fully peer-to-peer are able to provide stronger guarantees
  of forward progress \cite{paxos}. Since we do not handle traditional fault
  tolerance or network churn,  this is an area
  for future work \todo{lulz}
  \paragraph{Active attacks:}
  Global traffic analysis attacks only require the adversary to be able to
  monitor global traffic. If the adversary can also modify or generate
  traffic, several other attacks are possible. The adversary can launch
  \emph{man-in-the-middle} (MITM) attacks in which it impersonates Alicia,
  Badru, or one or more Tor nodes. The adversary can launch \emph{Sybil}
  attacks, in which many different Tor clients controlled by the same
  adversary join the network as individual clients. Either of these can be
  used to create \emph{Denial-of-Service} (DoS) attacks, which might either
  prevent users from connecting to Tor at all, or force Tor traffic to go
  through particular, potentially adversary-controlled, Tor nodes ---
  de-anonymizing the users.
