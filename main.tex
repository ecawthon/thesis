% !TEX TS-program = pdflatex
% !TEX encoding = UTF-8 Unicode

% Based on Pomona's template here:
% http://www.cs.pomona.edu/classes/senior-exercise/2011/
% also based on Yale's papers repository makefile setup

% \documentclass[draftcopy,short]{srpaper}
\documentclass[finalcopy,nolof,nolot]{srpaper}
\usepackage[colorinlistoftodos, color=blue!20]{todonotes}
% \usepackage[disable]{todonotes}
\usepackage[pdftex,pdfusetitle]{hyperref}
\usepackage{setspace}
\usepackage{multirow}
\usepackage{tabularx}
\usepackage{adjustbox}
\usepackage{amsthm}
\usepackage{amsfonts}
\setlength{\marginparwidth}{2.5cm}

% Custom todos
\newcommand{\todogrunt}[1]{\todo[caption={#1}]
  {\begin{spacing}{1.0}TODO: #1\end{spacing}}}
\newcommand{\todoword}[1]{\todo[color=green!40,caption={#1}]
  {\begin{spacing}{1.0}TODO: #1\end{spacing}}}
\newcommand{\todosubst}[1]{\todo[color=orange,caption={#1}]
  {\begin{spacing}{1.0}TODO: #1\end{spacing}}}
\newcommand{\tocite}{\todo[color=yellow]{TODO: cite}}
\newcommand{\toadd}[1]{\todo[inline,caption={#1}]
  {\begin{spacing}{1.0}TODO: #1\end{spacing}}}
\newcommand{\note}[1]{\todo[inline,color=red!20,caption={#1}]
  {\begin{spacing}{1.0}#1\end{spacing}}}

\input macros.tex

\setcounter{tocdepth}{2}

% theorem environments
\newtheorem{theorem}{Theorem}
\newtheorem{lemma}[theorem]{Lemma}

\title{Decentralized Group Management in Dissent}
\author{Eleanor Cawthon}
\date{\today, 2015}
\advisor{Professors Bryan Ford and Tzu-Yi Chen, advisors}
\abstract{
Among both humans and computers, decentralized approaches to group
decision-making exhibit trade-offs between decentralization and scalability.

We provide two contributions: First, we outline a general specification for
election protocols providing instigator anonymity. Next, we sketch how this can
be applied to provide bootstrapping and group management for the Dissent
anonymity protocol.

% \todogrunt{finish rewriting}

% TODO: Make this the intro for the protocol section.
We go on to present a protocol for egalitarian
groups to determine their leadership in a fashion that is anonymous, verifiable,
and fully decentralized.  By combining the Dissent protocol for anonymous
communication with decentralized trust\cite{p2pd}, with a simple voting protocol
utilizing linkable ring signatures\cite{lrs}, we show how a group might attain
verifiable and anonymous elections with Byzantine trust assumptions, secure
against a global passive adversary.  As a specific example, we show how this
might be used as a server selection and group management mechanism in scalable
Dissent \cite{din}, so that the scalable protocol is used most of the time, but
where the peer-to-peer consensus can always rescind the power it has delegated.

}
\acknowledgment{}

% hyperref setup, from template
\hypersetup{
           plainpages=false,
           pageanchor=true,
           breaklinks=true,
           bookmarkstype=toc,
           bookmarksopenlevel=2,
           bookmarksnumbered=true,
           colorlinks=true,
           linkcolor=black,
           urlcolor=magenta,
           citecolor=black
         }

\begin{document}
\frontmatter

\chapter{Introduction}\label{Chapter:Intro}
\toadd{In \GrassContext, \Org~was established as an equalizing
  force: \Marginalized s and \Dominant s alike could share in governance, with
  one vote per person. The \BadGuys, however, could see the transcripts of all
  meetings of \Org. If a \Marginalized~made a proposal unpopular with \BadGuys,
  the \Marginalized~would be \Disappeared; if a \Dominant~made the same
proposal, \BadGuys~could do nothing.}
Or,
\toadd{
  In \HistContext, \Tyrant~legally became \Office. Over the next several years,
\Congress~passed a series of electoral reforms that ultimately allowed for
\Axis~ control of \Congress~and granted \Tyrant~complete, authoritarian control
of \Domain.  \Tyrant~established \MartialLaw, and it took \War~to reverse this.
Of course, \Tyrant~did have a great deal of popular support, at least initially.
He promised \Platform. As the saying goes, Mussolini made the trains run on
time. But whether or not a majority of the people of \Domain~ever supported
\Axis, by the time \Congress~had given them control, the people could not revoke
his power without \War.
}

This is one example of a more general problem in group decision-making:
% In very
% small groups, action can proceed by consensus - all members have the opportunity
% to be heard, and only actions that have the support of the entire group proceed.
% In any moderately sized group, however, this peer-to-peer approach to consensus
% becomes unweildly. Most governance structures implement some sort of delegation
% of power \tocite, whether by way of an elected legislature or a military
% dictator.
%
In order to maintain the advantages of democracy and decentralized
trust, while also allowing efficient governance at scale, it must be possible to
delegate power to individuals or small groups, and it must be possible to revoke
that power.

Various computational approaches exist to providing secure and trustworthy
democratic elections, where each individual's vote is anonymous and where the
result is verifiable. As the case of \Tyrant shows, however, securing the
election process itself is not enough when a central power determines whether
and what elections should be held \todosubst{maybe communist elections are a
better example here}.
Noam Chomsky writes that
``[t]he smart way to keep people passive and obedient is to strictly limit the
spectrum of acceptable opinion, but allow very lively debate within that
spectrum'' \cite{chomsky1998common}. To have truly free elections, the means for
calling for a vote and drafting the ballot must also be decentralized and secure
against coercion. We shall henceforth refer to this process of initiating a vote
as a \emph{petition}. If \Congress had a mechanism for anonymously petitioning
for a vote of no confidence, for example, it might have been possible to
determine support for removing \Tyrant without endangering the instigator of
that vote and without resorting to \War.
% \note{I don't think this matters right now}
% Further, voting over the internet poses additional trust problems beyond those
% inherent to electronic voting in general.  A dissident group organizing in
% defiance of a powerful entity with control over the network must protect its
% members' anonymity not only from other group members, but from a global passive
% adversary who can analyze all message transmissions and traffic patterns among
% all nodes in the system.
%

\todogrunt{Probably could be better/fix the rest of this section for real
outline}
This paper makes three contributions. First, we we provide what we believe to be
a novel examination of anonymity in the context of electronic voting. Next, we
propose a specification and implementation sketch for a verifiable, anonymous,
and decentralized petition protocol. Finally, we show how this can be applied to
provide group management in the Dissent in Numbers\tocite anonymity protocol,
with applications for scalable anonymous web browsing.

We begin with an overview of existing tools dealing with various aspects of this
problem (Chapter~\ref{Chapter:Existing}).  We then outline the properties
provided by our protocol (Chapter~\ref{Chapter:Goals}), and specify the threat
model against which it is secure (Chapter~\ref{Chapter:Threats}).  In
Chapter~\ref{Chapter:Spec}, we provide a detailed specification for a protocol
providing the properties laid out in Chapter~\ref{Chapter:Goals}. In
Chapter~\ref{Chapter:Protocol}, we outline one potential implementation of this
specification\todo{TODO: proofs section?}. Finally, we conclude and discuss
directions for future work (Chapter~\ref{Chapter:Conclusion}).\todoword{fewer
``outline''}

\chapter{Goals}\label{Chapter:Goals}
\input goals.tex

\chapter{Related Work}\label{Chapter:Existing}
\input litreview.tex

\chapter{Threat Model}\label{Chapter:Threats}
\input threatmodel.tex

\chapter{General Specification}\label{Chapter:Spec}
\input spec.tex

\chapter{Protocol Description}\label{Chapter:Protocol}
\input protocol.tex

\chapter{Conclusion}\label{Chapter:Conclusion}
We have argued that verifiability is not enough for a
post-Snowden electronic voting protocol. We have specified and analyzed the
properties a voting protocol would need to provide in order to conform to a
revised trust model, wherein peers are not required to trust one another, and
wherein the adversary is assumed to be able to monitor all network traffic. We
have further contributed the first voting protocol we know of that provides the
verifiability guarantees of the electronic voting literature, the strong
anonymity of Dissent, and the fully decentralized trust model of Byzantine
peer-to-peer systems.

Future work will involve a more detailed specification and analysis of the
protocol sketched in Chapter~\ref{Chapter:Protocol}. Work has already begun on
implementing it and integrating the election of servers into future versions of
Dissent, utilizing the implementation of linkable ring signatures available in
Go \cite{golrs}. Once it is known to be secure, the changes should be fully
implemented and performance should be analyzed in the face of various kinds of
disruption, and also client churn.

As networked communication becomes increasingly integral to the communications
of humans in collectives, questions of security too become essential to analysis
of voting systems.  We believe this will be useful in constructing systems for
use by activists and governments alike as these infrastructures develop.

\listoftodos

% \chapter{Annotated Bibliography}\annotatedbibliography{annotated}
\bibliographystyle{alpha}
\bibliography{annotated}
% \appendix
% \chapter{Why Intermediate Vote Counts Can't Be
% Secret}\label{Appendix:SecretProof}
% \input secret.tex

\end{document}
