% !TEX TS-program = pdflatex
% !TEX encoding = UTF-8 Unicode

% Based on Pomona's template here:
% http://www.cs.pomona.edu/classes/senior-exercise/2011/
% also based on Yale's papers repository makefile setup

% \documentclass[draftcopy,short]{srpaper}
\documentclass[finalcopy,nolof,nolot]{srpaper}
% \usepackage[colorinlistoftodos, color=blue!20]{todonotes}
\usepackage[disable]{todonotes}
\usepackage[pdftex,pdfusetitle]{hyperref}
\usepackage{setspace}
\usepackage{multirow}
\usepackage{tabularx}
\usepackage{adjustbox}
\usepackage{amsthm}
\setlength{\marginparwidth}{2.5cm}

% Custom todos
\newcommand{\todogrunt}[1]{\todo[caption={#1}]
  {\begin{spacing}{1.0}TODO: #1\end{spacing}}}
\newcommand{\todoword}[1]{\todo[color=green!40,caption={#1}]
  {\begin{spacing}{1.0}TODO: #1\end{spacing}}}
\newcommand{\todosubst}[1]{\todo[color=orange,caption={#1}]
  {\begin{spacing}{1.0}TODO: #1\end{spacing}}}
\newcommand{\tocite}{\todo[color=yellow]{TODO: cite}}
\newcommand{\toadd}[1]{\todo[inline,caption={#1}]
  {\begin{spacing}{1.0}TODO: #1\end{spacing}}}
\newcommand{\note}[1]{\todo[inline,color=red!20,caption={#1}]
  {\begin{spacing}{1.0}#1\end{spacing}}}
\setcounter{tocdepth}{2}

% theorem environments
\newtheorem{theorem}{Theorem}
\newtheorem{lemma}[theorem]{Lemma}

\title{Decentralized Group Management in Dissent}
\author{Eleanor Cawthon}
\date{\today}
\advisor{Professors Bryan Ford and Tzu-Yi Chen, advisors}
\abstract{
Among both humans and computers, decentralized approaches to group
decision-making exhibit trade-offs between decentralization and scalability.

We provide two contributions: First, we outline a general specification for
election protocols providing instigator anonymity. Next, we sketch how this can
be applied to provide bootstrapping and group management for the Dissent
anonymity protocol.

% \todogrunt{finish rewriting}

% TODO: Make this the intro for the protocol section.
We go on to present a protocol for egalitarian
groups to determine their leadership in a fashion that is anonymous, verifiable,
and fully decentralized.  By combining the Dissent protocol for anonymous
communication with decentralized trust\cite{p2pd}, with a simple voting protocol
utilizing linkable ring signatures\cite{lrs}, we show how a group might attain
verifiable and anonymous elections with Byzantine trust assumptions, secure
against a global passive adversary.  As a specific example, we show how this
might be used as a server selection and group management mechanism in scalable
Dissent \cite{din}, so that the scalable protocol is used most of the time, but
where the peer-to-peer consensus can always rescind the power it has delegated.

}
\acknowledgment{}

% hyperref setup, from template
\hypersetup{
           plainpages=false,
           pageanchor=true,
           breaklinks=true,
           bookmarkstype=toc,
           bookmarksopenlevel=2,
           bookmarksnumbered=true,
           colorlinks=true,
           linkcolor=black,
           urlcolor=magenta,
           citecolor=black
         }

\begin{document}
\frontmatter

\chapter{Introduction}\label{Chapter:Intro}
A classic problem in human group interaction is how to make decisions in a way
so that everyone is represented, but progress is still made. In very small
groups, action can proceed by consensus - all members have the opportunity to be
heard, and only actions that have the support of the entire group proceed. In
any moderately sized group, however, this peer-to-peer approach to consensus
becomes unweildly. Most governance structures implement some sort of delegation
of power \tocite, whether by way of an elected legislature or a military
dictator.

Although this has traditionally been characterized as a problem of communication
at scale, we can also conceptualize it as a problem of trust. Participants in a
democratic group place some amount of trust in the will of the consensus, but
wish to avoid trusting any individual or small group with enough power for them
to usurp the democratic process.

How, then, might such a group minimize the risk of assigning enough power to a
small group for that group to misbehave, while maximizing the economies of scale
arising from delegating power?

\todosubst{Not sure we need this paragraph, because this isn't the tradeoff
  we're solving.}
The electoral process itself is a particularly interesting example of this
phenomenon. Elections frequently utilize secret ballots in order to prevent
voters from being coerced into voting a particular way\tocite, but these schemes
traditionally involve trusting a centralized entity to honestly count the votes.
In contrast, election mechanisms that allow voters to verify their votes have
been counted correctly, such as a vote by role call or raise of hand, typically
sacrifice ballot secrecy\tocite.

Various computational approaches exist to addressing these limitations of
traditional elections. The trust considerations decentralized groups face,
however, extend beyond the voting process itself. Noam Chomsky writes that
``[t]he smart way to keep people passive and obedient is to strictly limit the
spectrum of acceptable opinion, but allow very lively debate within that
spectrum'' \cite{chomsky1998common}. To have truly free elections, the means for
calling an election and for drafting the ballot must also be decentralized. For
example, in \tocite many systems, there is a mechanism for calling a vote of no
confidence\todoword{capitalize or hyphenate?} to potentially remove elected
leaders in the middle of a term. Existing electronic voting protocols do not
provide a way of protecting the identity of the voter who decides such a
referendum should take place.
% \note{I don't think this matters right now}
% Further, voting over the internet poses additional trust problems beyond those
% inherent to electronic voting in general.  A dissident group organizing in
% defiance of a powerful entity with control over the network must protect its
% members' anonymity not only from other group members, but from a global passive
% adversary who can analyze all message transmissions and traffic patterns among
% all nodes in the system.
%
\todogrunt{Probably could be better/fix the rest of this section for real
outline}
We present a survey and analysis of secure, decentralized, consensus-based
decision making in untrusted networks. We provide what we believe to be a novel
examination of anonymity in the context of electronic voting, and propose a
general specification for verifiable, anonymous, and decentralized group
management.

We begin with an overview of existing tools dealing with various aspects of this
problem (Chapter~\ref{Chapter:Existing}).  We then outline important properties
for voting protocols operating in this trust model, and introduce several
important threat models to consider (Chapter~\ref{Chapter:Goals}).  In
Chapter~\ref{Chapter:Spec}, we provide a detailed specification for a protocol
providing the properties laid out in Chapter~\ref{Chapter:Goals}. In
Chapter~\ref{Chapter:Protocol}, we outline one potential implementation of this
specification.\todo{TODO: proofs section?}. Finally, we conclude and discuss
directions for future work (Chapter~\ref{Chapter:Conclusion}).

\chapter{Related Work}\label{Chapter:Existing}
\input litreview.tex

\chapter{Goals}\label{Chapter:Goals}
\input goals.tex

\chapter{Threat Model}\label{Chapter:Threats}
\input threatmodel.tex

\chapter{General Specification}\label{Chapter:Spec}
\input spec.tex

\chapter{Protocol Description}\label{Chapter:Protocol}
\input protocol.tex

\chapter{Conclusion}\label{Chapter:Conclusion}
We have argued that verifiability is not enough for a
post-Snowden electronic voting protocol. We have specified and analyzed the
properties a voting protocol would need to provide in order to conform to a
revised trust model, wherein peers are not required to trust one another, and
wherein the adversary is assumed to be able to monitor all network traffic. We
have further contributed the first voting protocol we know of that provides the
verifiability guarantees of the electronic voting literature, the strong
anonymity of Dissent, and the fully decentralized trust model of Byzantine
peer-to-peer systems.

Future work will involve a more detailed specification and analysis of the
protocol sketched in Chapter~\ref{Chapter:Protocol}. Work has already begun on
implementing it and integrating the election of servers into future versions of
Dissent, utilizing the implementation of linkable ring signatures available in
Go \cite{golrs}. Once it is known to be secure, the changes should be fully
implemented and performance should be analyzed in the face of various kinds of
disruption, and also client churn.

As networked communication becomes increasingly integral to the communications
of humans in collectives, questions of security too become essential to analysis
of voting systems.  We believe this will be useful in constructing systems for
use by activists and governments alike as these infrastructures develop.

\listoftodos

% \chapter{Annotated Bibliography}\annotatedbibliography{annotated}
\bibliographystyle{alpha}
\bibliography{annotated}
% \appendix
% \chapter{Why Intermediate Vote Counts Can't Be
% Secret}\label{Appendix:SecretProof}
% \input secret.tex

\end{document}
