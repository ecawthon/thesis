In this chapter, we outline a general specification for verifiable, anonymous,
and fully decentralized petition protocols. For simplicity, we assume all votes
have only two options (ratify or not), but note that it can be proven that any
other multiple choice ballots can be constructed from these components and thus
our analysis is fully generalizable.

The protocol involves two layers: An anonymous broadcast channel whereby
participants can pass arbitrary messages, and the petition protocol which
utilizes this to construct petitions.
\section{Anonymous Broadcast Protocol}
We assume an anonymous communication layer that provides the following
functionality:

\subsection{Definitions}
\begin{itemize}
\item An instance of the anonymous communication layer consists of:

\begin{itemize}
\item A fixed peer set
  $\SetPeers:=\peer_{1},\peer_{2},...,\peer_{\NumClients}.$
\item A pseudonym scheme $\FunNym:\SetPeers\longleftrightarrow\SetNymDomain$,
  where $\SetNymDomain$ is a set of pseudonyms and $\FunNym$ is a bijection
\item A monotonically increasing turn number $\turn$, associated with a
  (possibly blank) message $\msg_{\turn}$ in the broadcast channel signed by
  pseudonym $\FunNym(\peer_{i})\in \SetNymDomain$.
\item A \KwSchedule~function $\FunSched:\mathbb{N}\to \SetNymDomain$ mapping
  rounds to pseudonyms, establishing whose turn it is.
\end{itemize}
\end{itemize}

\subsection{Interface}

Peers have the following functions available to them:
\begin{itemize}
\item At any round $\turn$, Any peer $\peer_{i}$can pseudononymously broadcast
a message $\msg_{\NumClients(\peer_{i})}$ to $\SetPeers$ at round $\turn+j$
for some $j\le \NumClients$
\item At the end of round $\turn$, all $\peer\in \SetPeers$ learn the contents
  of $\msg_{\turn}$
\end{itemize}

\section{Anonymous Petition Protocol}
\subsection{Data Structures}
  A \KwManifest~defines the group configuration and consists of
  \begin{itemize}
    \item A \KwRoster, representing the set of eligible voters
    \item A function $\FunEval : \SetElectionStates \to (\SetResult, \SetZKPs)$
      where $\SetElectionStates$ is the set of all \emph{Election State}s, and
      $\SetZKPs$ is the set of all possible proofs of correctness of the result.
      That is, if for some outcome $\ElectionState$, we
      have$\FunEval(\ElectionState) = (\Choice, \VarZKP)$, then the ballot
      choice $\Choice \in \ElectionState.\VarPetition.\SetChoices$ passes.  A
      plausible example of such a $\FunEval$ is the function which specifies
      what proportion of \KwPeer s must agree to a change in the composition of
      the \KwRoster~in order for the change to take effect.
    \item Any other group configuration information the group should be able to
      vote on.
  \end{itemize}

  A \StructPetition~is a proposal for the \KwCluster to vote on. It consists of
    \begin{itemize}
      \item An \StructInstigator, the \KwPeer~who proposed the
        Petition. This information is not publicly associated with
        the Petition,
      % \item A unique identifier $\VarLinkScope$, which is public
      % \item A proposed new \KwManifest,
      \item A proposal text $\msg$,
      \item A set of ballot choices $\SetChoices$, and
      \item An expiration condition\todogrunt{express in terms of states?},
        defining when the vote on the petition should end.
    \end{itemize}

  A \StructBallot~encodes information about the eligibility of the voter,
    and information about the voter's preference.
    \footnote{To determine the results of
    the elction while providing the verifiability properties discussed in
    Section~\ref{Section:verif}, there must be a public record of some
    aggregate information about each: An auditor must be able to tell that every
    voter was eligible, and also what the outcome of the election was. To
    provide voter confidentiality, we must provide a way for each voter to
    provide both bits of information without exposing the correlation between
    the two. In other words, if Badru wants to vote for Alicia to be president,
    Badru must convey that Badru (or someone with Badru's credentials) voted,
    and that a vote has been cast for Alicia, without revealing that Badru cast
    a vote for Alicia. We can represent the information Badru provides as a
    \StructBallot~tuple $(\sig, \vote)$, where $\sig$ encodes Badru's
    credentials and $\vote$ encodes his candidate choice.

    To provide the necessary information while preserving his
    confidentiality, Badru must encrypt part or all of his ballot. It is
    impossible to design a performant and Byzantine fault tolerant protocol
    where both are kept secret. (I have discovered a truly marvelous proof of
    this, which this margin is too narrow to contain \tocite). This leaves two
    possibilities: Either Badru can encode his credentials in a $\sig$ that is
    anonymous (c.f. \cite{lrs}) or he can encrypt $\vote$ so that Badru's choice
    of candidates can only be decyphered in aggregate, once the connection to
    Badru's public signature has been lost (as in Dissent).
  }


\subsection{States}
\begin{itemize}
  \item A \StructState~is a tuple $(\VarManifest, \round)$ defining the current
    cluster, where $\VarManifest$ is a \KwManifest, and $\round$ is a
    monotonically increasing unique state identifier (e.g., a logical
    clock)\tocite.

  \item An \StructElection~encompasses the operation of the protocol between
    when a Petition $\VarPetition$ is first proposed and when the expiration
    condition is met --- that is, the portion of the protocol where members are
    aware that that $\VarPetition$ is being considered, but in which no member
    knows the outcome of the election.

  \item The \StructElectionState~can be described by a tuple $(\State,
    \VarPetition, \SetVotes)$, where $\State$ is the current \StructState,
    $\VarPetition$ is the Petition being voted on, and $\SetVotes$ is a
    collection of \StructBallot s that have been cast thus far.

\end{itemize}

\subsection{Functions}
Within finite time and in a way that is fair, every \KwPeer~should be able
to call each of the following functions:
\begin{itemize}
  \item \NamePropose(\StructPetition~$\VarPetition$): broadcasts $\VarPetition$
    to the cluster and initiates an \StructElection.
  \item \NameVote$_{\StructState~\State}$(\StructPetition~$\VarPetition$), which
    constructs a \StructBallot~$\Ballot$ and broadcasts it to the cluster in
    conjunction with $\VarPetition$
  \item \NameEvaluate$_{\StructState~\State}$(\StructPetition~$\VarPetition$,
    \SetVotes) $\to (\{\AtomUnfinished, \AtomValid, \AtomInvalid\}, \Choice,
    \VarZKP)$:
    Given an election state, every peer should be able to determine whether the
    election has ended, and if it has, what the result was. If it cannot, or if
    any part of the election state is invalid, it should be able to provide a
    $\VarZKP$ of misbehavior.
  \item \NameEvaluate$_{\StructState~\State}$(\StructPetition~$\VarPetition$,
    \StructBallot~$\Ballot$) $\to$ (\{\AtomTrue, \AtomFalse\}, $\VarZKP$) should
    return (\AtomTrue, $\VarZKP$) if $\Ballot$ was produced by a valid \KwPeer~
    according to $\State$, and is a vote on $\VarPetition$. Otherwise,
    \AtomFalse~and a proof of misbehavior should be produced.
\end{itemize}

