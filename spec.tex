In this chapter, we outline a general specification for verifiable, anonymous,
and fully decentralized petition protocols. For simplicity, we assume all votes
have only two options (ratify or not), but note that it can be proven that any
other multiple choice ballots can be constructed from these components and thus
our analysis is fully generalizable.

The protocol involves two layers: An anonymous broadcast channel whereby
participants can pass arbitrary messages, and the petition protocol which
utilizes this to construct petitions.
\section{Anonymous Broadcast Protocol}
We assume an anonymous communication layer that provides the following
functionality:

\subsection{Definitions}
\begin{itemize}
\item An instance of the anonymous communication layer consists of:

\begin{itemize}
\item A fixed peer set
  $\SetPeers:=\peer_{1},\peer_{2},...,\peer_{\NumClients}.$
\item A pseudonym scheme $\FunNym:\SetPeers\longleftrightarrow\SetNymDomain$,
  where $\SetNymDomain$ is a set of pseudonyms and $\FunNym$ is a bijection
\item A monotonically increasing turn number $\turn$, associated with a
  (possibly blank) message $\msg_{\turn}$ in the broadcast channel signed by
  pseudonym $\FunNym(\peer_{i})\in \SetNymDomain$.
\item A \KwSchedule~function $\FunSched:\mathbb{N}\to \SetNymDomain$ mapping
  rounds to pseudonyms, establishing whose turn it is.
\end{itemize}
\end{itemize}

\subsection{Interface}

Peers have the following functions available to them:
\begin{itemize}
\item At any round $\turn$, Any peer $\peer_{i}$can pseudononymously broadcast
a message $\msg_{\NumClients(\peer_{i})}$ to $\SetPeers$ at round $\turn+j$
for some $j\le \NumClients$
\item At the end of round $\turn$, all $\peer\in \SetPeers$ learn the contents
  of $\msg_{\turn}$
\end{itemize}

\section{Anonymous Petition Protocol}
\subsection{Data Structures}
\begin{itemize}
  \item This protocol operates at the level of a \KwCluster~of
    \KwNode s.  A \KwPeer~is a \KwNode~that is a member of a
    particular \KwCluster. It may be voted out of a cluster. New nodes
    become \KwPeer s if their joining is approved by the existing cluster.

  \item A \KwManifest~defines the group configuration and consists of
  \begin{itemize}
    \item A \KwRoster, representing the set of eligible voters
    \item A map \texttt{Committees} mapping names of elected statuses to sets of
      elements of the \KwRoster. For example, \texttt{Committees} might
      consist of the key ``\texttt{servers}'', with the value $\{A, B, C\}$,
      where $A, B,$ and $C$ are the participants listed in $R$ which have been
      elected to act as \texttt{servers} in an instance of Dissent in
      Numbers.\todosubst{Obsolete. Make generic.}
    \item A function $\FunEval : \SetElectionStates \to (\SetResult, \SetZKPs)$
      where $\SetElectionStates$ is the set of all \emph{Election State}s, and
      $\SetZKPs$ is the set of all possible proofs of correctness of the result.
      That is, if $\FunEval(\ElectionState) = \AtomTrue$ for some outcome
      $\ElectionState$, then the proposal corresponding to $\ElectionState$
      passes.  A plausible example is the function which specifies what
      proportion of \KwPeer s must agree to a change in the composition of the
      \KwRoster~in order for the change to take effect.
  \end{itemize}
  \item A \StructPetition~is a proposal to change the \KwManifest. It
    consists of
    \begin{itemize}
      \item An \StructInstigator, the \KwPeer~who proposed the
        Petition. This information is not publicly associated with
        the Petition.
      \item A unique identifier $L$, which is public
      \item A proposed new \KwManifest,
      \item An expiration condition, defining when the vote on the petition
        should end.
    \end{itemize}
  \item A \StructBallot~encodes information about the eligibility of the voter,
    and information about the voter's preference. To determine the results of
    the elction while providing the verifiability properties discussed in
    Section~\ref{Section:verif}, there must be a public record of some
    aggregate information about each: An auditor must be able to tell that every
    voter was eligible, and also what the outcome of the election was. To
    provide voter confidentiality, we must provide a way for each voter to
    provide both bits of information without exposing the correlation between
    the two. In other words, if Badru wants to vote for Alicia to be president,
    Badru must convey that Badru (or someone with Badru's credentials) voted,
    and that a vote has been cast for Alicia, without revealing that Badru cast
    a vote for Alicia. We can represent the information Badru provides as a
    \StructBallot~tuple $(\sig, \vote)$, where $\sig$ encodes Badru's
    credentials and $\vote$ encodes his candidate choice.

    \note{this is kindof out of no where. maybe move?}
    To provide the necessary information while preserving his
    confidentiality, Badru must encrypt part or all of his ballot.
    We show
    in Appendix~\ref{Appendix:SecretProof}\todosubst{I think this is actually
    only true if people vote at different times} that it is impossible to design
    a Byzantine Fault Tolerant protocol where both are kept secret. This leaves
    two possibilities:
    Either Badru can encode his credentials in a $\sig$ that is anonymous
    (c.f. \cite{lrs})
    or he can encrypt $\vote$ so that Badru's choice of candidates can only be
    decyphered in aggregate, once the connection to Badru's public signature has
    been lost.

\end{itemize}

\subsection{States}
\begin{itemize}
  \item A \StructState~is a tuple $(\VarManifest, \round)$ defining the current
    cluster, where $\VarManifest$ is a \KwManifest, and $\round$ is a
    monotonically increasing unique state identifier (i.e., a logical
    clock)\tocite.

  \item An \StructElection~encompasses the operation of the protocol between
    when a Petition $\VarPetition$ is first proposed and the round
    $\VarPetition$.\round~--- that is, the portion of the protocol where
    members are aware that that $\VarPetition$ is being considered, but in which
    no member knows the outcome of the election.

  \item The \StructElectionState~can be described by a tuple $(\State,
    \VarPetition, \SetVotes)$, where $\State$ is the current \StructState,
    $\VarPetition$ is the Petition being voted on, and $\SetVotes$ is a
    collection of \StructBallot s that have been cast thus far.

\end{itemize}

\subsection{Functions}
Within finite time and in a way that is fair, every \KwPeer~should be able
to call each of the following functions:
\begin{itemize}
  \item \NamePropose(\StructPetition~$\VarPetition$): broadcasts $\VarPetition$
    to the cluster and initiates an \StructElection
  \item \NameVote$_{\StructState~\State}$(\StructPetition~$\VarPetition$), which
    constructs a \StructBallot~$\Ballot$ and broadcasts it to the cluster in
    conjunction with $\VarPetition$
  \item \NameEvaluate$_{\StructState~\State}$(\StructPetition~$\VarPetition$,
    \StructBallot~$\Ballot$) $\to (\{\AtomUnfinished, \AtomValid,
    \AtomInvalid\}, \VarZKP$: Given an election state, every peer should be able
    to determine whether the election has ended, and if it has, what the result
    was. If it cannot, or if any part of the election state is invalid, it
    should be able to provide a $\VarZKP$ of misbehavior.
  \item \NameEvaluate$_{\StructState~\State}$(\StructPetition~$\VarPetition$,
    \StructBallot~$\Ballot$) $\to$ (\{\AtomTrue, \AtomFalse\}, $\VarZKP$) should
    return (\AtomTrue, $\VarZKP$) if $\Ballot$ was produced by a valid \KwPeer~
    according to $\State$, and is a vote on $\VarPetition$. Otherwise,
    \AtomTrue~and a proof of misbehavior should be produced.
\end{itemize}

