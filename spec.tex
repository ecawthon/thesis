In this chapter, we outline a general specification for verifiable, anonymous,
and fully decentralized petition protocols. For simplicity, we assume all votes
have only two options (ratify or not), but note that it can be proven that any
other multiple choice ballots can be constructed from these components and thus
our analysis is fully generalizable.

The protocol involves two layers: An anonymous broadcast channel whereby
participants can pass arbitrary messages, and the petition protocol which
utilizes this to construct petitions.
\section{Anonymous Broadcast Protocol}
We assume an anonymous communication layer that provides the following
functionality:

\subsection{Definitions}
An instance of the anonymous communication layer consists of:
\begin{itemize}
\item A fixed peer set
  $\SetPeers:=\peer_{1},\peer_{2},...,\peer_{\NumClients}.$
\item A pseudonym scheme $\FunNym:\SetPeers\longleftrightarrow\SetNymDomain$,
  where $\SetNymDomain$ is a set of pseudonyms and $\FunNym$ is a bijection
\item A monotonically increasing turn number $\turn$, associated with a
  (possibly blank) message $\msg_{\turn}$ in the broadcast channel signed by
  pseudonym $\FunNym(\peer_{i})\in \SetNymDomain$.
\item A \KwSchedule~function $\FunSched:\mathbb{N}\to \SetNymDomain$ mapping
  rounds to pseudonyms, establishing whose turn it is. Each client should have
  their $i^{th}$ term before any client has its $i+1^{th}$ turn.
\item A \HistoryOracle~, which all clients can query to learn the messages
  transmitted in previous rounds.
\end{itemize}

\subsection{Interface}

Peers have the following functions available to them in polynomial time:
\begin{itemize}
  \item \NameReceive($\turn, \peer$) $\to (\nym, \msg)$ allows $\peer$ to query
    the \HistoryOracle~and returns the $\turn^{th}$ message along with its owner
    $\nym$ (with $\FunSched(\turn) = \nym $)
  \item \NameSend($\turn, \peer, \msg$) --- if called at turn $\turn$, the peer
    $\peer$ broadcasts its message $\msg$ at the next turn where
    $\FunSched(\turn+i) = p$, wiith $i \le \NumClients$. Equivalently, the
    \HistoryOracle~is updated such that, after turn $\turn+i$, any peer calling
    \NameReceive($\turn+i$) will retrieve $(\FunNym(\peer), \msg)$.
\end{itemize}

\subsection{Security Properties}

\section{Anonymous Petition Protocol}
\subsection{Data Structures}
  A \KwManifest~defines the group configuration and consists of
  \begin{itemize}
    \item A \KwRoster, representing the set of eligible voters
    \item A function $\FunEval : \SetElectionStates \to (\SetResult, \SetZKPs)$
      where $\SetElectionStates$ is the set of all \emph{Election State}s, and
      $\SetZKPs$ is the set of all possible proofs of correctness of the result.
      That is, if for some outcome $\ElectionState$, we
      have$\FunEval(\ElectionState) = (\Choice, \VarZKP)$, then the ballot
      choice $\Choice \in \ElectionState.\VarPetition.\SetChoices$ passes.  A
      plausible example of such a $\FunEval$ is the function which specifies
      what proportion of \KwPeer s must agree to a change in the composition of
      the \KwRoster~in order for the change to take effect.
    \item Any other group configuration information the group should be able to
      vote on.
  \end{itemize}

  A \StructPetition~is a proposal for the \KwCluster to vote on. It consists of
    \begin{itemize}
      \item An \StructInstigator, the \KwPeer~who proposed the
        Petition. This information is not publicly associated with
        the Petition,
      % \item A unique identifier $\VarLinkScope$, which is public
      % \item A proposed new \KwManifest,
      \item A proposal text $\msg$,
      \item A set of ballot choices $\SetChoices$, and
      \item An expiration condition $\FunExpir : \cdot \to \{\AtomTrue,
        \AtomFalse\}$ defining when the vote on the petition should end.
   \end{itemize}

  A \StructBallot~encodes information about the eligibility of the voter,
    and information about the voter's preference.
    \footnote{To determine the results of
    the elction while providing the verifiability properties discussed in
    Section~\ref{Section:verif}, there must be a public record of some
    aggregate information about each: An auditor must be able to tell that every
    voter was eligible, and also what the outcome of the election was. To
    provide voter confidentiality, we must provide a way for each voter to
    provide both bits of information without exposing the correlation between
    the two. In other words, if Badru wants to vote for Alicia to be president,
    Badru must convey that Badru (or someone with Badru's credentials) voted,
    and that a vote has been cast for Alicia, without revealing that Badru cast
    a vote for Alicia. We can represent the information Badru provides as a
    \StructBallot~tuple $(\sig, \vote)$, where $\sig$ encodes Badru's
    credentials and $\vote$ encodes his candidate choice.

    To provide the necessary information while preserving his
    confidentiality, Badru must encrypt part or all of his ballot. It is
    impossible to design a performant and Byzantine fault tolerant protocol
    where both are kept secret. (I have discovered a truly marvelous proof of
    this, which this margin is too narrow to contain \tocite). This leaves two
    possibilities: Either Badru can encode his credentials in a $\sig$ that is
    anonymous (c.f. \cite{lrs}) or he can encrypt $\vote$ so that Badru's choice
    of candidates can only be decyphered in aggregate, once the connection to
    Badru's public signature has been lost (as in Dissent).
  }


% \subsection{States}
% \begin{itemize}
%   \item A \StructState~is a tuple $(\VarManifest, \round)$ defining the
%   current cluster, where $\VarManifest$ is a \KwManifest, and $\round$ is a
%   monotonically increasing unique state identifier (e.g., a logical
%   clock)\tocite.
%
%   \item An \StructElection~encompasses the operation of the protocol between
%   when a Petition $\VarPetition$ is first proposed and when the expiration
%   condition is met --- that is, the portion of the protocol where members are
%   aware that that $\VarPetition$ is being considered, but in which no member
%   knows the outcome of the election.
%
%   \item The \StructElectionState~can be described by a tuple $(\State,
%     \VarPetition, \SetVotes)$, where $\State$ is the current \StructState,
%     $\VarPetition$ is the Petition being voted on, and $\SetVotes$ is a
%     collection of \StructBallot s that have been cast thus far.
%
% \end{itemize}
%
\subsection{Interface}
Within finite time and in a way that is fair, every \KwPeer~should be able
to call each of the following functions:
\begin{itemize}
  \item \NamePropose($\round, \peer, \VarPetition$):
    whereby a \KwPeer $\peer$ constructs a \StructPetition~$\VarPetition$ and
    \NameSend s it to the cluster for consideration
  \item \NameVote($\round, \peer, \VarManifest, \VarPetition, \Ballot$): If
    $\VarPetition.\FunExpir(\cdot) = \AtomFalse$, then a \KwPeer~\peer may
    \NameVote on it by creating a \StructBallot~\Ballot indicating its preferred
    outcome, and \NameSend ing this to the group.
  \item \NameEvaluate($\round, \peer, \VarManifest, \VarPetition, \SetVotes) \to
    (\{\AtomUnfinished, \AtomValid, \AtomInvalid\}, \Choice, \VarZKP)$: Given a
    vote state, every peer should be able to determine whether the vote
    has ended, and if it has, what the result was. If it cannot, or if any part
    of the election state is invalid, it should be able to provide a $\VarZKP$
    of misbehavior.
  \item \NameEvaluate($\round, \peer, \VarManifest, \VarPetition, \SetVotes$)
    $\to$ (\{\AtomTrue, \AtomFalse\}, $\VarZKP$) should return (\AtomTrue,
    $\VarZKP$) if $\Ballot$ was produced by a valid \KwPeer~according to
    $\State$, and is a vote on $\VarPetition$. Otherwise, \AtomFalse~and a proof
    of misbehavior should be produced.
\end{itemize}

\subsection{Security Properties}
  \subsubsection{Verifiability}
  \paragraph{Individual}
  A group management protocol provides \emph{individual verifiability} if, in
  any Election State $(\State, \VarPetition, \SetVotes)$, any member
  $\VarMember$ either knows its own vote is correctly represented in $\SetVotes$
  (that is, either $\VarMember$ voted and $\VarMember$'s signature for
  $\VarPetition.\VarLinkScope$ \todo{TODO: more params; define lrs} is contained
  in $\SetVotes$, or $\VarMember$ did not vote and $\VarMember$'s signature is
  absent from $\SetVotes$), or can produce a zero-knowledge proof that the
  Election State is invalid.\todo{TODO: define invalid}

  \paragraph{Universal}
  A group management protocol provides \emph{universal verifiability} if, in any
  finished election state, $(\State, \VarPetition, \SetVotes)$ , anybody can
  verify that $\SetVotes$ is a valid signing of $\VarPetition$ or else produce a
  proof that it is not. Consequently, any auditor (member or otherwise) can
  verify the canonical value of $\VarManifest.\FunEval(\ElectionState)$.

  \paragraph{Eligibility}
  A group management protocol provides \emph{eligibility verifiability} if, in
  any finished election state, anybody can verify that all elements of
  $\SetVotes$ were cast by \KwMember s of the current cluster.

  \subsubsection{Anonymity}
  \paragraph{For Instigators}
  A group management protocol provides \emph{instigator anonymity} if, during
  and after any election, no member and no outside observer can determine
  which member proposed the ballot in question.

  \paragraph{Of Ballots}
  A group management protocol provides \emph{secret ballots} if, during and
  after any election, either no outside observer can reconstruct which member
  submitted which vote, or no outside observer can reconstruct how any
  member voted. The same restrictions apply to knowledge gained by other
  participants, except that each member can trivially reconstruct its own vote.

