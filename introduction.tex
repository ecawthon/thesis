% Wherever dissidents seek to organize in defiance of an oppressor, they must
% balance the need to communicate effectively with the need to work around their
% adversary's capacity to disrupt their group.  In particular, the problem of
% trust is ubiquitous \todo{cite}.
% We present a protocol designed to allow groups of peers to communicate
% efficiently, anonymously, and securely.
%
A classic problem in human group interaction is how to make decisions in a way
so that everyone is represented, but progress is still made. In very small
groups, action can proceed by consensus - all members have the opportunity to be
heard, and only actions that have the support of the entire group proceed. In
any moderately sized group, however, this peer-to-peer approach to consensus
becomes unweildly. Most governance structures implement some sort of delegation
of power \tocite, whether by way of an elected legislature or a military
dictator.

Although this has traditionally been characterized as a problem of communication
at scale, we can also conceptualize it as a problem of trust. Participants in a
democratic group place some amount of trust in the will of the consensus, but
wish to avoid trusting any individual or small group with enough power for them
to usurp the democratic process.

How, then, might such a group minimize the risk of assigning enough power to a
small group for that group to misbehave, while maximizing the economies of scale
arising from delegating power?

We present a protocol for group communication that provides both of these
properties. By combining the Dissent in Numbers protocol for anonymous
communication with decentralized trust, with a simple voting protocol utilizing
linkable ring signatures, we show how a group might attain accountable
scalability: where the scalable protocol is used most of the time, but where
the peer-to-peer consensus can always rescind the power it has
delegated.

As a motivating example, \todogrunt{Describe something like the workers for
justice example}

Our protocol could help this group.\todogrunt{Describe how they could use it}

\todo[inline]{TODO: roadmap}

