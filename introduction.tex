% Wherever dissidents seek to organize in defiance of an oppressor, they must
% balance the need to communicate effectively with the need to work around their
% adversary's capacity to disrupt their group.  In particular, the problem of
% trust is ubiquitous \todo{cite}.
% We present a protocol designed to allow groups of peers to communicate
% efficiently, anonymously, and securely.
%
A classic problem in human group interaction is how to make decisions in a way
so that everyone is represented, but progress is still made. In very small
groups, action can proceed by consensus - all members have the opportunity to be
heard, and only actions that have the support of the entire group proceed. In
any moderately sized group, however, this peer-to-peer approach to consensus
becomes unweildly. Most governance structures implement some sort of delegation
of power \tocite, whether by way of an elected legislature or a military
dictator.

Although this has traditionally been characterized as a problem of communication
at scale, we can also conceptualize it as a problem of trust. Participants in a
democratic group place some amount of trust in the will of the consensus, but
wish to avoid trusting any individual or small group with enough power for them
to usurp the democratic process.

How, then, might such a group minimize the risk of assigning enough power to a
small group for that group to misbehave, while maximizing the economies of scale
arising from delegating power?

The electoral process itself is a particularly interesting example of this
phenomenon. Elections frequently utilize secret ballots in order to prevent
voters from being coerced into voting a particular way\tocite, but these schemes
traditionally involve trusting a centralized entity to honestly count the votes.
In contrast, election mechanisms that allow voters to verify their votes have
been counted correctly, such as a vote by role call or raise of hand, typically
sacrifice ballot secrecy\tocite. \todoword{transition sentence?}

Various computational approaches exist to addressing these limitations of
traditional elections. The trust considerations decentralized groups face,
however, extend beyond the voting process itself. Noam Chomsky writes that
``[t]he smart way to keep people passive and obedient is to strictly limit the
spectrum of acceptable opinion, but allow very lively debate within that
spectrum'' \cite{chomsky1998common}. To have truly free elections, the means for
calling an election and for drafting the ballot must also be decentralized. For
example, in \tocite many systems, there is a mechanism for calling a vote of no
confidence\todoword{capitalize or hyphenate?} to potentially remove elected
leaders in the middle of a term. Existing electronic voting protocols do not
provide a way of protecting the identity of the voter who decides such a
referrendum should take place. Further, voting over the internet poses
additional trust problems beyond those inherent to electronic voting in general.
A dissident group organizing in defiance of a powerful entity with control over
the network must protect its members' anonyity not only from other group
members, but from a global passive adversary who can analyze all message
transmissions and traffic patterns among all nodes in the system.

We present a protocol for egalitarian groups to determine their leadership in a
fashion that is anonymous, verifiable, and fully decentralized.  By combining
the Dissent protocol for anonymous communication with decentralized
trust\cite{p2pd}, with a simple voting protocol utilizing linkable ring
signatures\cite{lrs}, we show how a group might attain verifiable and anonymous
elections with Byzantine trust assumptions, secure against a global passive
adversary.  As a specific example, we show how this might be used as a server
selection and group management mechanism in scalable Dissent \cite{din}, so that
the scalable protocol is used most of the time, but where the peer-to-peer
consensus can always rescind the power it has delegated.

Chapter~\ref{Chapter:Background} provides an overview of existing work on
decentralized decisionmaking, with particular attention to anonymity and
electronic voting systems, before situating our contribution with discussion of
these systems' limitations. Chapter~\ref{Chapter:Goals} outlines the
functionality and security properties our protocol sets out to provide.
Chapter~\ref{Chapter:Protocol} describes our protocol specification.
Chapter~\ref{Chapter:Proofs} formalizes the properties set out in
Chapter~\ref{Chapter:Goals} and includes proofs that our protocol
satisfies them. Chapter~\ref{Chapter:Conclusion} concludes.\todoword{update for
actual org}
% The major contribution of this work is to provide a protocol in which not only
% are elections verifiable, but the decision to have an election and the content
% of the ballots can be determined by any member at any time,
% anonymously.\todo{highlight this paragraph but rewrite it because I'm
% tired\ldots}
%
% Pluralism: More than one valid opinion is possible
%
% Networked: Adversary model is ridiculous
%
% Decentralized: We don't trust anyone.
%
